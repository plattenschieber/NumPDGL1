\documentclass[12pt]{scrbook}

\usepackage{geometry}
\geometry{a4paper,left=2.5cm,right=2.5cm, top=2cm, bottom=3cm, marginpar=2cm} 

%% Sonderzeichen
\usepackage[ngerman]{babel}
\usepackage[utf8]{inputenc}
\usepackage[T1]{fontenc}
\usepackage{amsmath,amssymb,amsfonts,stmaryrd,mathtools,amsthm,esint}
\usepackage{algpseudocode}
\usepackage{dsfont} % bb für Zahlen
%\usepackage{MnSymbol} %adobe minion font

%% minted
%\usepackage{minted}
%\newminted{cpp}{linenos}

%\usepackage{marginnote}
%\renewcommand*{\marginfont}{\footnotesize} 

\usepackage{graphicx}
%\graphicspath{{Bilder/}}
%\usepackage{tikz}
%\usepackage[all]{xy}
%\usetikzlibrary{arrows,calc,shadows,patterns,through,backgrounds}
%\pgfdeclarepatternformonly{mynwlines}{%
%\pgfqpoint{-1pt}{-1pt}}{\pgfqpoint{8pt}{8pt}}{\pgfqpoint{6pt}{6pt}}%
%{
%  \pgfsetlinewidth{0.4pt}
%  \pgfpathmoveto{\pgfqpoint{0pt}{3pt}}
%  \pgfpathlineto{\pgfqpoint{6.1pt}{-3.1pt}}
%  \pgfusepath{stroke}
%}

\usepackage{hyperref}
\usepackage{makeidx}
\usepackage{listings}
\usepackage{enumerate}

%% Counters
%\renewcommand\theequation{\thesection.\arabic{equation}}
%\AtBeginSection[]{ \setcounter{section}{0} }
\numberwithin{equation}{section}


\newcommand{\N}{\mathbb{N}}
\newcommand{\Z}{\mathbb{Z}}
\newcommand{\R}{\mathbb{R}}
\newcommand{\C}{\mathbb{C}}
\newcommand{\F}{\mathbb{F}}
\newcommand\K{\mathbb{K}}
\renewcommand{\c}[1]{\mathcal{#1}} % beliebiger Buchstabe kann mit \op in cal geschrieben werden 
\renewcommand\P{\mathcal{P}} % tolles P, anstatt eines Zeilenumbruchzeichens 
\newcommand\Ac{\mathcal{A}}
\newcommand\Cc{\mathcal{C}}
\newcommand\Gc{\mathcal{G}}
\newcommand\Pc{\mathcal{P}}
\newcommand\Lc{\mathcal{L}}
\newcommand\Kc{\mathcal{K}}
\newcommand\Mc{\mathcal{M}}
\newcommand\Tc{\mathcal{T}}
\newcommand\LC{\mathcal{L}}
\newcommand\ii{\mathrm{i}} % schlichtes i, ungeschwungen
\newcommand{\ud}{\,\textnormal{d}} % aufrechtes d fürs Integral
\newcommand{\floor}[1]{\left\lfloor #1 \right\rfloor}
\newcommand{\ceil}[1]{\left\lceil #1 \right\rceil}
%\def\default{}
\renewcommand{\emph}[2][\default]{\ifx#1\default\textbf{#2}\index{#2}\else\textbf{#2}\index{#1}\fi}
%\newcommand{\ocirc}{\, \raisebox{1pt}{\footnotesize\textcircled{$\circ $}}\,}
%\newcommand{\ostar}{\, \raisebox{1pt}{\footnotesize\textcircled{$*$}}\,}

\newcommand{\bs}[1]{\boldsymbol{#1}} % fett gedruckt
\newcommand{\entspr}{\mathop{\widehat{=}}} % entspricht Zeichen
\newcommand{\eps}{\varepsilon} % schönes epsilon 
\def\pre{\textnormal{pre}}
\def\suc{\textnormal{suc}}

% coole Abkürzungen für die Norm und Doppelnorm
\newcommand{\norm}[1]{\|#1\|}
\newcommand{\znorm}[1]{|\!|\!|#1|\!|\!|}

% ein paar Befehle in Normalschrift, statt kursiv
\newcommand\dist{\textnormal{dist}}
\newcommand\im{\textnormal{Im}\,}
\newcommand\re{\textnormal{Re}\,}
\newcommand\ran{\textnormal{Ran}\,}
\renewcommand\ker{\textnormal{Ker}\,}
\newcommand\range{\textnormal{range}\,}
\newcommand\var{\textnormal{Var}}
\newcommand\cov{\textnormal{Cov}}
\newcommand\ggt{\textnormal{ggT}}
\newcommand\periode{\textnormal{Periode}}
\newcommand{\trace}{\textnormal{trace}\,}
\newcommand{\sign}{\textnormal{sign}\,}
\newcommand{\cond}{\textnormal{cond}}
\newcommand{\spanf}{\textnormal{span}\,}
\newcommand{\spann}{\textnormal{span}}
\newcommand{\grad}{\textnormal{grad}\,}
\newcommand{\sgn}{\textnormal{sgn}}
\newcommand{\rd}{\textnormal{rd}}
\newcommand{\diag}{\textnormal{diag}}
\newcommand{\blockdiag}{\textnormal{blockdiag}}
\newcommand{\supp}{\textnormal{supp}}
\newcommand{\fie}{\varphi}
\newcommand{\eins}{\mathds{1}}
\newcommand{\diam}{\mathrm{diam}}
\newcommand{\const}{\mathrm{const}}
\newcommand{\divt}{\mathrm{div}}
\newcommand{\vol}{\mathrm{vol}}
\newcommand{\inn}{\textnormal{ in }}
\newcommand{\auf}{\textnormal{ auf }}
\renewcommand{\i}{\textnormal{i}}
\def\dtilde{\stackrel{\approx}}
\def\dabs{\phantom{a}\quad}


\newcommand{\tn}{\ensuremath{| \! | \! |}}
\newcommand{\operateson}{\rcirclearrowright}
\usepackage{framed,color}                          % Farbe
\setlength{\fboxsep}{0.2cm}
\setlength{\fboxrule}{1pt}
\definecolor{shadecolor}{rgb}{0.91,0.91,1}      % fuer shaded-Umgebungen

\newtheoremstyle{note}% name
  {1ex}  % Space above
  {1ex}  % Space below
  {\sl}  % Body font
  {}     % Indent amount (empty = no indent, \parindent = para indent)
  {\bfseries}  % Thm head font
  {.}    % Punctuation after thm head
  {.5em} % Space after thm head: " " = normal interword space;
         % \newline = linebreak
  {\thmname{#1}\thmnumber{ #2}\bf\thmnote{(#3)}}     % Thm head spec (can be left empty, meaning `normal')

\newtheoremstyle{remark}% name
  {1ex}  % Space above
  {1ex}  % Space below
  {}     % Body font
  {}     % Indent amount (empty = no indent, \parindent = para indent)
  {\bfseries}  % Thm head font
  {.}    % Punctuation after thm head
  {.5em} % Space after thm head: " " = normal interword space;
         % \newline = linebreak
  {\thmname{#1}\thmnumber{ #2}\bf\thmnote{(#3)}}     % Thm head spec (can be left empty, meaning `normal')

\theoremstyle{note}
\newtheorem{amssatz}{Satz}[section]
\newtheorem{amslemma}[amssatz]{Lemma}
\newtheorem{amsproblem}[amssatz]{Problem}
\newtheorem{amsdefi}[amssatz]{Definition}
\theoremstyle{remark}
\newtheorem{amsbsp}[amssatz]{Beispiel}
\newtheorem*{amsbsp*}{Beispiel}
\newtheorem{amsalgo}[amssatz]{Algorithmus}
\newtheorem*{amsalgo*}{Algorithmus}
\newtheorem{amskorollar}[amssatz]{Korollar}
\newtheorem{amsbemerkung}[amssatz]{Bemerkung}
\newtheorem*{amsbemerkung*}{Bemerkung}

\newenvironment{satz}[1][]{\begin{amssatz}[#1]\begin{shaded}}
                          {\end{shaded}\end{amssatz}}
\newenvironment{lemma}[1][]{\begin{amslemma}[#1]\begin{shaded}}
                           {\end{shaded}\end{amslemma}}
\newenvironment{problem}[1][]{\begin{amsproblem}[#1]\begin{shaded}}
                           {\end{shaded}\end{amsproblem}}
\newenvironment{korollar}[1][]{\begin{amskorollar}[#1]\begin{shaded}}
                           {\end{shaded}\end{amskorollar}}
\newenvironment{algorithmus}[1][]{\begin{amsalgo}[#1]\begin{shaded}}
                                 {\end{shaded}\end{amsalgo}}
\newenvironment{algorithmus*}[1][]{\begin{amsalgo*}[#1]\begin{shaded}}
                                 {\end{shaded}\end{amsalgo*}}
\newenvironment{definition}[1][]{\begin{amsdefi}[#1]\begin{shaded}}
                                {\end{shaded}\end{amsdefi}}
\newenvironment{beispiel}[1][]{\begin{amsbsp}[#1]\begin{framed}}
                               {\end{framed}\end{amsbsp}}
\newenvironment{beispiel*}[1][]{\begin{amsbsp*}[#1]}
                               {\end{amsbsp*}}
\newenvironment{bemerkung}[1][]{\begin{amsbemerkung}[#1]\begin{framed}}
                               {\end{framed}\end{amsbemerkung}}
\newenvironment{bemerkung*}[1][]{\begin{amsbemerkung*}[#1]\begin{framed}}
                               {\end{framed}\end{amsbemerkung*}}




\hypersetup{
    bookmarks=true,         % show bookmarks bar?
    unicode=false,          % non-Latin characters in Acrobat’s bookmarks
    pdftoolbar=true,        % show Acrobat’s toolbar?
    pdfmenubar=true,        % show Acrobat’s menu?
    pdffitwindow=true,      % page fit to window when opened
    pdftitle={Numerik PDG I},    % title
    pdfauthor={Jeronim Morina},     % author
    pdfsubject={Skript zur Vorlesung im SS 2014},   % subject of the document
    pdfnewwindow=true,      % links in new window
    pdfkeywords={},         % list of keywords
    colorlinks=true,        % false: boxed links; true: colored links
    linkcolor=red,          % color of internal links
    citecolor=green,        % color of links to bibliography
    filecolor=magenta,      % color of file links
    urlcolor=cyan           % color of external links
}
\makeindex
\begin{document}
\begin{titlepage}
\vspace*{\stretch{1}}
\flushright{\Huge\bfseries Numerik partieller Differentialgleichungen I}
\noindent\rule[-1ex]{\textwidth}{4pt}\\[5pt]
\Large Vorlesungsskript SS 2014 \\[3cm]
{\Large \bf Vorlesungsmitschrift von Jeronim Morina}\\[5mm]
\hfill \textnormal{\date{}}
\vspace{\stretch{1}}
\end{titlepage}

\clearpage
\thispagestyle{empty}
\pagenumbering{roman} \setcounter{page}{1}

\tableofcontents

\chapter*{Vorwort}
Dieses Dokument enthält die Mitschrift von Jeronim Morina zur Vorlesung ``Numerik partieller Differentialgleichungen I'' im Somersemester 2014 bei Professor Kunoth. Wir können dem Leser weder Vollständigkeit noch Fehlerfreiheit (von dieser sind wir überzeugt, dass sie definitiv nicht gegeben ist) versprechen. Wir sind jedoch für Verbesserungsvorschläge dankbar, diese können an \href{mailto:morina@jeronim.de}{morina@jeronim.de} geschickt werden.\\


\hfill Köln, \today

\cleardoublepage
\pagenumbering{arabic} \setcounter{page}{1}

\section*{Grundlegende Definitionen}
Sei $\Omega \subset \R^d$ ein Gebiet (d.h. offen und zusammenhängend), $d\in\N$.
\begin{eqnarray*}
     x:=(x_1, \cdots, x_d)^T\in\R^d\\
     \alpha\in\Z_+^d \text{Multiindex}, |\alpha| := \alpha_1+\cdots+\alpha_d, x^\alpha:=x_1^{\alpha_1} \cdots x_d^{\alpha_d}\\
    \C^k(\Omega) := \left\{ f:\Omega\to\R: \frac{\partial^{|\alpha|}f}{\partial x^\alpha} \in \C^0(\Omega), \forall |\alpha|\leq k \right\}
\end{eqnarray*}

\begin{definition}%1.1.1
    Jede Funktion $u\in\C^k(\Omega)\times\C^l(\R)$, die zu gegebenem 
     $G: \R^{d+1}\times \R^{d+k \choose d)}\times\R^l \to \R^m$ die Gleichung 
     $G\left(x,t, \frac{\partial^{|\alpha|}u}{\partial x^\alpha}: |\alpha| \leq k, \frac{\partial^i u}{\partial t^i}, 1\leq i\leq l\right) = 0$(1.1.2) erfüllt, heißt klassische Lösung der durch (1.1.2) gegebenen PDE der Ordnung k im Ort und l in der Zeit.
\end{definition}

$d+k \choose d$: die Anzahl der Möglichkeiten aller Ableitungen

\begin{itemize}
    \item ``klassisch'': Lösung wird als genügend glatt vorausgesetzt
    \item ``partiell'': Differentialgleichung enthält partielle Ableitungen
    \item $k,l$: Ordnung der PDE bzgl. Ort und Zeit
\end{itemize}

\begin{beispiel}
    $\underset{(Raumdim)}{d=2}, \underset{(skalare Gleichung)}{m=1}, \underset{(Ordn. Ort)}{k=1}, \underset{(keine Zeitabl.)}{l=0}$
    \[
        G\left( x_1,x_2,u,\frac{\partial u}{\partial x_1}, \frac{\partial u}{\partial x_2} \right) := a(x) \frac{\partial u}{\partial x_1} + b(x) \frac{ \partial u}{\partial x_2} + c(x) u = 0
    \]
    Gesucht: $u=u(x_1,x_2), \tilde(u) = \tilde(u)(x_1,x_2,t) = u(x_1,x_2)v(t)$
\end{beispiel}

Beachte: wie bei ODEs benötigt man Nebenbedingungen, um aus Gesamtlösungsschar für (1.1.2) eine spezielle Lösung \underline{eindeutig} auszuwählen.

\begin*{definition}[Problem ``korekt gestellt'']
\begin{itemize}
    \item es gibt eind. Lösung
    \item Lösung hängt stetig von Nebenbedingungen/Daten ab (Anfangs-, Randbedingungen)
\end{itemize}

Definition (1.1.1) ist viel zu allgemein für WissRech. Dynamik der am häufigsten auftretenden physikalischen Prozesse aus zwei Klassen: 
\begin{enumerate}
    \item[(A)] Diffusion (Wärmeleitung, Black-Scholes-Gleichung)
    \item[(B)] Transport (Strömungsprobleme)
\end{enumerate}
\end{definition}
Für Numerik zunächst: lineare Prozesse oder linearisierte Prozesse (da nichtlineare PDEs gewöhnlicherweise quadratisch sind und in einem kleinen Gebiet linear approximiert werden können).

Weitere Klassifikationen:
\begin{enumerate}[(a)]
    \item irreversible Prozesse (Reaktion, Diffusion, Wärmeleitung)
    \item reversible Prozesse (Wellen)
    \item stationäre Prozesse (Variationsprobleme, Statik [Beispiel Stadion])
\end{enumerate}

Faustregel für mathematiche Sichtweise:
\begin{enumerate}[(a) $\to$]
    \item parabolische PDEs
    \item hyperbolische PDEs
    \item elliptische PDEs
\end{enumerate}

Wichtig für WissRech: Prozesstyp entscheidet \underline{Wahl} der Diskretisierungsmethode zur Entwicklung effizienter Algorithmen zur Berechnung einer approximativen Lösung.
Prinzipiell möglich: Differenzenverfahren (auf uniformen Gitter)
Probleme dabei: 
\begin{itemize}
    \item benötigt hohe Regularität der Lösung (Taylorentwicklung)
        \item of nicht physikalisch sinnvoll, z.B. Transportprobleme $\to$ Finite-Volumen-Verfahren
\end{itemize}

\section{Klassifizierung linearer PDEs 2.Ordnung im $\R^2$}
Betrachte auf $\Omega\subset\R^2$ Gebiet 
\begin{equation} %1.2.1
    L[u] := A u_{xxx} + Bu_{xy} + Cu_{yy} + E_1 u_x + E_2 u_y + E_3 u = F
    \label{eq:PDE}
\end{equation}

(z.B. Laplace mit $A=B=-1$)%anmerkung
gesucht $u=u(x,y)=u(x_1,x_2), L:C^2(\Omega)\to C^0(\Omega) $
(Wenn wir zwei mal stetig diff'bare Funktionen zwei mal Ableiten sollten wir in $C^0$ landen)

Gegeben: $A,B,C,E_i\in C^1(\Omega)$ Koeffizientenfunktionen (können von $x,y,u_x,u_y,u$ abhängen).
\eqref{eq:PDE} heißt: 

\begin{itemize}
        \item PDEear, wenn $L$ linear in den höchsten Ableitungen ist
        \item linear, falls $A,B,C$ nur von $x,y$ abhängen und in
            \begin{equation} %1.2.1a
                E:= E_1 u_x + E_2 u_y + E_3 u - F
            \end{equation}
            die $E_i$ auch nur von $x,y,$ abhängen
        \item linear mit konstanten Koeffizienten, falls $A,B,C,E$ konstant
        \item homogen, falls $F\equiv 0$
\end{itemize}

Der Term $Au_{xx} + Bu_{xy} + Cu_{yy}$ heißt Hauptteil der PDE. \\

Annahme: $A,B,C$ nicht gleichzeitg identisch $0$ sind (Funktionen!) auf $\Omega$
\underline{Frage} Existenz und Eindeutigkeit von (lokalen) Lösungen von \eqref{eq:PDE} unter Vorgabe von \underline{Anfangswerten} auf einer Kurve $\Cc$? 


\underline{Grundidee}
$\Cc \subset y$-Achse (Bild 1)
\begin{equation} %1.2.2
    u(0,y) = f(y), \quad y\in[a,b] \text{ mit } f\in C^{\infty}(\Omega) \text{ gegeben.}
    \label{}
\end{equation}
$\to$ dann auch $\frac{\partial}{\partial y} u(0,y), \frac{\partial^2}{\partial y^2} u(0,y),\cdots \quad y\in [a,b]$ bekannt.\\
Ist nun zudem bekannt: 
\[
    \left( \frac{\partial}{\partial x} u \right) (0,y) = g(y), \quad y\in [a,b], g\in C^{\infty}(\Omega) \text{ geg.}
\]
dann lassen sich auch 
\[
    \left( \frac{\partial^2}{\partial x \partial y} u\right)(0,y), 
    \left( \frac{\partial^3}{\partial y^2 \partial x} u \right)u(0,y), \cdots
\]
bestimmen.\\

Falls $A(0,y) \neq 0 \forall y\in \Cc$ (relativ starke Forderung an alle Punkte, jedoch beim Standardbsp. Laplace erfüllt  $A=-1$ überall diese Bedingung) liefert \eqref{eq:PDE} $u_{xx}(0,y):$
\[
    \underset{\neq 0}{A(0,y)} \underset{?}{u_{xx}(0,y)} + B(0,y) \underset{bekannt}{u_{xy}(0,y)} + C(0,y) \underset{bekannt}{u_{yy}(0,y)} + \underset{bekannt}{E(0,y)} = 0
\]

Differentiation von \eqref{eq:PDE} nach $x$ liefert $\frac{\partial^3}{\partial x^3}u(0,y)$, etc. (rekursive bestimmung der Ableitungen - wir haben $u$ hiermit immer noch nicht berechnet!!)

Fazit: Ist $u$ und $u_x$ auf $C$ bekannt $\Rightarrow$ auf der Kurve $C$ lassen sich alle partiellen Ableitungen von $u$ berechnen.

Mit Taylorentwicklung erhalten wir als Lösung 
\begin{equation} %1.2.4
    u(x,y) = u(0,y) + x \left( \frac{\partial u}{\partial x} \right)(0,y) + \frac{x^2}{2} \left( \frac{\partial^2 u}{\partial x^2} \right) (0,y) + \cdots
    \label{eq:taylor}
\end{equation}
(Diese Entwicklung konvergiert nicht notwendigerweise! - nur wenn sie konvergiert erhalten wir auch eine Lösung)
falls \underline{Konvergenz} der Taylorreihe sichergestellt werden kann.

\begin{satz}[Cauchy-Kowalewskaya (Speziellfall $\Cc\subset y$-Achse)]
    Falls $\frac{B}{A}, \frac{C}{A}, \frac{E}{A}, f, g$ in einer Umgebung von $\Cc$ (reell) analytisch sind,  \underline{dann} konvergiert die Taylorreihe \eqref{eq:taylor}
\end{satz}

Nach Konstruktion ist dann \eqref{eq:PDE} mit Anfangsbedingung (1.22) $u(0,y) = f(y)$, (1.2.3) $u_x(0,y) = g(y), y\in\Cc$ \underline{eindeutig lösbar} (lokal).

Allgemein: Man untersucht die Lösbarkeit von \eqref{eq:PDE} unter Vorgabe allgemeiner Anfangswerte auf beliebiger Kurve $\Cc\subset \R^2$ (Bild2) 

Gegeben seien Anfangswerte: 
\begin{equation}[Cauchy-Daten] % als Kästchen dran
    u|_\Cc = f \\
    \underbrace{\frac{\partial u}{ \partial \nu}}_{\nu \nabla u (\nu \text{ innere Normale})} \big|_\Cc = g
\end{equation}

Da mit $u$ auch Ableitungen von $u$ entlang der Kurve $\Cc$ bekannt sind, kann man (versuchen erneut die Ableitungen zu bestimmen)  $p:= \frac{\partial}{ \partial x} u, q:=\frac{\partial}{\partial y} u$ auf $\Cc$ bestimmen, denn:\\

\[
    \frac{\partial u}{\partial \nu} = \nu \cdot \nabla u, \frac{\partial u}{\partial \tau} = \tau \cdot \nabla u (\tau,\nu \text{ bekannt})
\]

$\to$ 2 Gleichungen für 2 Unbekannte $u_x (=p), u_y(=q)$\\

\underline{Frage}: Unter welchen Bedingungen legt dies zusammen mit PDE \eqref{eq:PDE} alle Ableitungen 2.Ordnung 
\begin{equation} %1.2.6
    r:=\frac{\partial^2 u}{\partial x^2}, s:= \frac{\partial^2 u}{\partial x\partial y}, t:=\frac{\partial^2 u}{\partial y^2}
    \label{}
\end{equation}
fest?\\

Dazu: Sei $\Cc = \left\{ (x(l), y(l)): l\in [0,T] \right\}$ natürliche Parametrisierung
\[
    \Rightarrow \frac{dP}{dl} = \frac{\partial P}{\partial x} \frac{\partial x}{\partial l} + \frac{\partial P}{\partial y} \frac{\partial y}{\partial l} = r \frac{dx}{dl} + s \frac{dy}{dl}\\
    \frac{dq}{dl} = \cdots = s \frac{dx}{dl} + \frac{dy}{dl}
\]

\section*{Recapture of 1.2 Classification of linear PDEs of 2nd order in $\R^2$}

On $\Omega\subset\R^2$
\begin{equation*}%1.2.1
    L[u] := \underbrace{Au_{xx} + Bu_{xy} + Cu_{yy}}_{\text{principal part of PDE}} + \cdots = F
    \label{}
\end{equation*}
assumption: not all of $A,B,C \equiv 0$

Existence and uniqueness of solutions of (local) solutions, provided that initial values on curve $\Cc\subset\bar{\Omega}$ are given.\\
$\Cc$ a general curve, given Cauchy data 
\begin{eqnarray}
    u\big|_\Cc = f\\
    \frac{\partial u}{\partial \nu} \big|_\Cc = g
    \label{}
\end{eqnarray}

$\tau, \nu$ linearly independent $\Rightarrow$ computation of $\nabla u = \begin{pmatrix} u_x\\ u_y\end{pmatrix}$ is possible at any point on $\Cc$.

\underline{Frage:} unter welchen Bedingungen legt dies zusammen mit \ref{1.2.1} alle Ableitungen 2. Ordnung 
\begin{equation} %(1.2.6)
    r:=u_{xx}, s:=u_{xy}, t:=u_{yy}
    
    \label{}
\end{equation}
Dazu sei $\Cc = \left\{ \left( x(l), y(l) \right): l\in [0,T] \right\}$ natürliche Parametrisierung bezüglich der Bogenlänge L
\[
    \underset{p:=u_x\\q:=u_y}{\Rightarrow} \frac{dp}{dl} = \frac{\partial p}{\partial x} \frac{dx}{dl} + \frac{\partial p}{\partial y}\frac{dy}{dl} 
    = \underset{=u_{xx}}{r} \frac{dx}{dl} + \underset{=u_{xy}} \frac{ dy}{dl}
\]
\[
    \underbrace{\frac{dq}{dl}}_{\text{bekannt}} = \frac{\partial q}{\partial x} \frac{dx}{dl} + \frac{\partial q}{\partial y} \frac{dy}{dl}
    = s \frac{dx}{dl} + t\frac{dy}{dl}
\]
Außerdem müssen $r,s,t$ die PDE \ref{1.2.1} erfüllen $\to$ lineares Gleichungssystem mit 3 Gleichungen für die 3 Unbekannten $r,s,t:$

\begin{eqnarray} %(1.2.7)
    A+\cdots + Bs + Ct &=&  -E\\
    \frac{dx}{dl} r + \frac{dy}{dl} s &=& \frac{dp}{dl}\\
    \frac{dx}{dl} s + \frac{dy}{dl} t &=& \frac{dq}{dl}
    
    \label{}
\end{eqnarray}

Cramersche Regel $\Rightarrow$
\begin{equation}
    r= \frac{\Delta_1}{\Delta_4}, s=\frac{\Delta_2}{\Delta_4}, t=\frac{\Delta_3}{\Delta_4}
    \label{}
\end{equation}

mit 
\[
    \Delta_1 :=
    \begin{vmatrix} -E & B & C \\
        \frac{dp}{dl} & \frac{dy}{dl} & 0 \\
        \frac{dq}{dl} & \frac{dx}{dl} & \frac{dy}{dl}
    \end{vmatrix}

    ,\Delta_2 :=
    \begin{vmatrix} A & -E & C \\
        \frac{dx}{dl} & \frac{dp}{dl} & 0 \\
        0 & \frac{dq}{dl} & \frac{dy}{dl}
    \end{vmatrix}

    ,\Delta_4 :=
    \begin{vmatrix} A & B & C \\
        \frac{dx}{dl} & \frac{dy}{dl} & 0 \\
        0 & \frac{dx}{dl} & \frac{dy}{dl}
    \end{vmatrix}
\]

$\to$ Eindeutige Lösung von \ref{1.2.7} $\Leftrightarrow$ $\Delta_4 \neq 0$ (für jeden Punkt auf $\Cc$)

\begin{bemerkung} %1.2.10
    Existenz und Eindeutigkeit einer Lösung von \ref{1.2.7} hängt \underline{nur} von $\Delta_4$ ab und erhält \underline{weder} Anfangsdaten $\left( \frac{dp}{dl}, \frac{dq}{dl}, p, q \text{ auf } \Cc \right)$ \underline{noch} die rechte Seite $-E$, sondern \underline{nur} Informationen über $A,B,C$ und geometrische Eigenschaften der Kurve.\\
\end{bemerkung}

Sind nun $u, p(=u_x), (=u_y), r(=u_{xx}), s(=u_{xy}), t(=u_{yy})$ auf $\Cc$ bekannt, differenziere \ref{1.2.1} bzgl. $y$. Wir erhalten dann:
\[
    A \underset{=u_{xxy}}{\frac{\partial r}{\partial y}} + B \frac{\partial s}{\partial y} + C \frac{\partial t}{\partial y} + \frac{\partial A}{\partial y} r + \frac{\partial B}{\partial y} s + \frac{\partial C}{\partial y} t = 0
\]

Mit $\tilde{u} := u_y, \tilde{r} := r_y, \tilde{s}:=s_y, \tilde{t}:=t_y, \tilde{p}:=\tilde{u}_y$ und $\tilde{E} \leftrightarrow E$ bekommt mit dem anderen beiden Gleichungen in \ref{1.2.7} Ausdrücke für $\tilde{r}=\frac{\tilde{\Delta_1}}{\tilde{\Delta_4}}, \tilde{s} = \frac{\tilde{\Delta_2}}{\tilde{\Delta_4}}\cdots$ ebenso für $\tilde{\tilde{u}} = u_x$ $\Rightarrow$ Ableitungen 3.Ordnung etc.

\underline{Insgesamt:} man erhält für $(x,y) \in \Cc$:
\begin{equation}%1.2.11
    u\left( x+d_x, y+d_y \right) = u(x,y) + (d_x) \underset{=u_x}{p} + (d_y)\underset{u_y}{q} + \frac{(d_x)^2}{2} \underset{u_{xx}}{r} + \frac{d_x d_y}{2} \underset{u_{xy}}{s} + \frac{(d_y)^2}{2} \underset{=u_{yy}}{t} + \text{ Terme höherer Ordnung}
    \label{}
\end{equation}

wobei $d_x, d_y$ klein. 

Falls die Reiehe konvergiert, so ist sie nach Konstruktion eindeutige Lösung von \ref{1.2.1} mit den Cauchy-Daten \ref{1.2.5}.

\begin{satz}[Cauchy-Kowalewskaya]
    Sind $A,B,C,E$ in einer Umgebung von $\Cc$ reell analytisch (d.h. können in eine Taylorreihe entwickelt werden, und stimmen auf dieser Umgebung punktweise überein), so hat \ref{1.2.1} mit den Cauchy-Daten \ref{1.2.5} in einer Umgebung von $\Cc$ genau dann eine eindeutige Lösung, wenn $\Delta_4 \neq 0$ gilt.
    
\end{satz}

Erinnere: Bedingung $\Delta_4 \neq 0$ beinhaltet nur Bedingungen an $A,B,C$ und an die Kurve $\Cc$.\\
Was ist bei $\Delta_4 = 0$? (Anm. $\Delta_1, \Delta_2, \Delta_3$ müssen auch null sein)
Um in diesem Fall überhaupt Lösungen für $r=u_{xx}, s=u_{xy}, t=u_{yy}$ zu erhalten, muss $\Delta_1 = \Delta_2 = \Delta_3 = 0$ gelten $\to \Delta_i = 0, i=1,2,3$ sind Kompatibilitätsbedingungen an die Anfangsdaten $p=u_x, q=u_y$. 

Anmerkung: Dabei verliert man allerdings die Eindeutigkeit. Wir gehen zunächst nicht weiter drauf ein.\\

$\Delta_4$ spielt eine besondere Rolle bei der Klassifizierung von PDEs \ref{1.2.2}:\\
betrachten $\Delta_4 = 0$ und entwickeln nach der 1. Zeile

\begin{eqnarray} %1.2.12

    0 = A \left( \frac{dy}{dl} \right)^2 - B \left( \frac{dx}{dl} \right) \left( \frac{dy}{dl} \right) + C \left( \frac{dx}{dl} \right)^2 &=&: Q\left( \frac{dy}{dl}, \frac{dx}{dl} \right) \quad :\left( \frac{dx}{dl} \right)^2\\
    \Leftrightarrow A \left( \frac{dy}{dx} \right)^2 - B \left( \frac{dy}{dx} \right) + C &=&  0 \quad \text{unabhängig von Parametrisierung}
\end{eqnarray}

$\Rightarrow$ charakteristisches Polynom des Hauptteils von PDE. Auch $Q(y,x) = Ay^2-Bxy +Cx^2$

\begin{definition}
    Kurevn $\Cc$, die die Bedingung \ref{1.2.12} bzgl. der PDE \ref{1.2.1} erfüllen, heißen Charakteristiken. Jede Richtung $\beta = (\beta_1,\beta_2)$, für die $Q(\beta)=0$, heißt charakteristische Richtung.
\end{definition}

%Bild Kurve
\underline{Beachte}: $\frac{dy}{dx}$ gibt die Steigung der Tangente an $\Cc$ in jedem Punkt an. Für Charakteristiken sind diese Steigungen Nullstellen des charakteristischen Polynoms. Ist Anfangskurve $\Cc$ charakteristisch, so liegt keine Eindeutigkeit der Lösung vor, da dann $\Delta_4 = 0$\\

Zur Klassifikation der PDE betrachte nun im Hinblick auf \ref{1.2.12} die Nullstellen des charakteristischen Polynoms $Q(z,1) := Az^2 - Bz + C = 0 (\Leftrightarrow \Delta_4 = 0) $

Die PDE \ref{1.2.1} heißt 
\begin{itemize}
    \item[hyperbolisch], falls $B^2-4AC > 0$, d.h. $Q(z,1)$ hat zwei verschiedene reelle Wurzeln $\to$, es existieren zwei verschiedene Scharen von Charakteristiken (d.h. zwei verschiedene Tangentensteigungen)
    \item[parabolisch], falls $B^2-4AC = 0$, d.h. $Q(z,1)$ hat eine doppelte Nullstelle, es existiert eine Schar von Charakteristiken
    \item[elliptisch], falls $B^2-4AC < 0$, d.h. es existieren keine reellen Nullstellen, also auch keine Charakteristiken.
\end{itemize}

\begin{beispiel}
    $-\Delta u =f \Leftrightarrow -(u_{xx} + u_{yy}) = -f$\\
    $\Rightarrow A=1, B=0, C=1 \Rightarrow -4 < 0 \Rightarrow $ elliptisch\\
    (Anmerkung: Irgendeine Kurve nehmen, Cauchy Daten darauf definieren. Bei elliptischen Problemen resultiert dies in einer eindeutig Lösung)
\end{beispiel}

\begin{itemize}
    \item Falls $Q(z,1)\left( =Az^2-Bz+C \right) = 0$ reelle Lösung hat, so sind die entsprechenden Charakteristiken durch \ref{1.2.13} $\frac{dy}{dx} = \frac{B \underset{+}{-} \sqrt{B^2-4AC}}{2A}$ gegeben.
\end{itemize}

'Charakteristikenschar': nur Steigung der Charakteristiken gegeben. (Anm. Bei einer hyperbolischen PDE dürfen z.B. keine Cauchy Daten auf den Charakteristiken (siehe Bild) vorgeben werden)



Jeder Kegelschnitt hat die Form 
\[
    ax^2 + 2bxy cy^2 + 2dx + 2dy + f = 0\\
    \delta:= \begin{vmatrix} a & b\\ b & c\end{vmatrix} = ac - b^2\\
\]
\begin{itemize}
    \item[$\delta>0 \Rightarrow$] Ellipse
    \item[$\delta=0 \Rightarrow$] Parabel
    \item[$\delta<0 \Rightarrow$] Hyperbel
\end{itemize}

\begin{bemerkung} %1.2.14
    \begin{enumerate}
        \item Falls $A=0, B\neq 0$: \eqref{1.2.12} erhält die Form 
            \[
                B\left( \frac{dy}{dx} \right) = C \Leftrightarrow \frac{dy}{dx} = \frac{C}{B} 
            \]
        \item Falls $A=0, C=0 \Rightarrow B\neq 0 \rightarrow \frac{dx}{dl}=0$ oder $\frac{dy}{dl}=0$ Charakteristiken parallel zum Koordinatensystem
    \end{enumerate}
\end{bemerkung}

\section*{Transformation von \eqref{eq:PDE} auf kanonische Form}

Da die Charakteristiken eine ausgezeichnete Rolle spielen, liegt es nahe, falls möglich (d.h. falls sie existieren), das Koordinatensystem danach auszurichten. (Anm. alte Koordinaten $x,y$: $(u=u(x,y))$)\\
Dazu: Seien $\xi: \R^2\to\R, \eta: \R^2\to\R$ und betrachte Kurvenscharen

\begin{align*}
    \Cc_\gamma :&=&  &\left\{ (x,y)\in\R^2: \xi(x,y)=\gamma \right\} \quad \text{(Höhenlinien, Niveaumengen)}\\
    \Gc_\gamma :&=&  &\left\{ (x,y)\in\R^2: \eta(x,y)=\gamma \right\} 
\end{align*}

mit 
\begin{equation} %1.2.15
    \begin{vmatrix}
        \xi_x & \xi_y\\
        \eta_x & \eta_y
    \end{vmatrix}
    = 
    \xi_x\eta_y - \xi_y\eta_x \neq 0 \quad \forall x,y
    \label{}
\end{equation}

d.h. $\nabla \xi \begin{pmatrix} \xi_x\\ \xi_y \end{pmatrix}, \nabla\eta$ linear unabhängig.

Wir erhalten zwei linear unabhängige Kurvenscharen, d.h. die Normalen $\nabla \xi, \nabla \eta$ sind linear unabhängig, oder 
\begin{align*}
    T:\R^2 &\to\R^2\\
    (x,y) &\to \left( \xi(x,y), \eta(x,y) \right) \quad \Cc^1-\text{Diffeomorphismus}\\
\end{align*}

\[
    \Rightarrow T' = 
    \begin{pmatrix}
        \xi_x & \xi_y \\
        \eta_x & \eta_y
    \end{pmatrix}
\]
mit $det T' \neq 0$.

Transformation von \eqref{eq:PDE}: Sei dazu $\Phi\left( \xi(x,y), \eta(x,y) \right):= u(x,y)$

\begin{align*}
    \underset{=\frac{\partial u}{\partial x}}{u_x} = \underbrace{\Phi_\xi \xi_x}_{vw} + \Phi_\eta \eta_x\\
    u_y = \Phi_\xi \xi_y + \Phi_\eta \eta_y\\
    \Rightarrow u_{xx} = \underbrace{\Phi_{\xi\xi}(\xi_x)^2 + \Phi_{\xi\eta} \xi_x\eta_x}_{v'w} + \Phi_\xi \underbrace{\xi_{xx}}_{vw'}
    + \Phi_{\eta\xi} \eta_x \xi_x + \Phi_{\eta\eta}(\eta_x)^2 + \Phi_\eta \eta_{xx}\\
    = \Phi_{\xi\xi} (\xi_x)^2 + \Phi_{\eta\eta}(\eta_x)^2 + 2\Phi_{\xi\eta} \eta_x \xi_x + \Phi_\xi \xi_{xx} + |phi_\eta \eta_{xx}\\
    u_{xy} = \Phi_{\xi\xi} \xi_x\xi_y + \Phi_{\eta\eta} \eta_y \eta_x + \Phi_{\xi\eta} (\xi_x\eta_y + \xi_y\eta_x) + \Phi_\xi \eta_{xy} + \Phi_\eta \eta_{xy}\\
    u_{yy} = \Phi_{\xi\xi}(\xi_y)^2 \Phi_{\eta\eta} (\eta_y)^2 + 2\Phi_{\xi\eta} \eta_y\xi_y + \Phi_\xi \xi_{yy} + \Phi_\eta \eta_{yy}
\end{align*} 

Also erhälit $L$ die Form (Koeffizientenvergleich, Übung)

\begin{equation} %1.2.16
    L[\Phi] = a \Phi_{\xi\xi} + b\Phi_{\xi\eta} + c \Phi_{\eta\eta} + \left( L[\xi] - E_3\xi \right)\Phi_\xi + \left( L[\eta] - E_3\eta \right) \Phi_\eta + E_3 Phi = F
    \label{eq:newPDE}
\end{equation}

mit \eqref{eq:newPDE}
\begin{align*}
    a:= A(\xi_x)^2 + B\xi_x\xi_y + C(\xi_y)^2\\
    b:= 2A\xi_x \eta_x + B(\xi_x\eta_y+\eta_x\xi_y) + 2C\xi_y\eta_y\\
    c:=A(\eta_x)^2 + B\eta_x\eta_y + C(\eta_y)^2
\end{align*}

(Anm. Durch die Transformation verlieren wir keine Informationen bzgl. der Klassifizierung)
Weiter gilt
\begin{equation}%1.2.17
    b^2-4ac = (B^2-4AC) = \underbrace{(\xi_x\eta_y-\xi_y\eta_x)^2}_{(det T')^2 > 0 \text{ nach Vor.}}
    \label{}
\end{equation}
d.h. $b^2-4ac$ und $B^2-4AC$ haben dasselbe Vorzeichen, d.h. die Klassifizierung (ellipt., parab., hyperb.) sind unabhängig von Koordinatentransformation \ref(1.2.14 (v))\\



Transformation der Cauchy-Daten \ref{1.2.5}
\[
    u\big|_\Cc = f, \Cc:=\Cc_0 = \left\{ (x,y): \xi(x,y)=0 \right\}
\]
$\to$ Daten $\Phi(0,\eta)$ ($\eta$-Achse), $\Phi_\eta(0,\eta), \Phi_{\eta\eta}(0,\eta)$ und $\frac{\partial u}{\partial \nu}\big|_\Cc = g$ (Normalenableitung)\\
$\Rightarrow \Phi_\xi(0,\eta), \Phi_{\xi\eta}(0,\eta)$ gegebe und weiter mit lokaler Taylorentwicklung wie im allgemeinen Satz von Cauchy-Kowalewskaya. \\

$L[\Phi]$ lässt sich nach $\Phi_{\xi\xi}$ auflösen, 
\begin{equation}
    a \underset{\text{Def.}}{=} Q(\xi_x,-\xi_y) \neq 0
    \label{}
\end{equation}

\begin{equation} %1.2.19
    \frac{dy}{dx} = - \frac{\xi_x}{\xi_y}
\end{equation}
$Q(\frac{dy}{dx},1) \neq 0$.\\

Damit: \underline{kanonische Formen von \eqref{eq:PDE}}
\begin{enumerate}[(i)]
    \item Sei PDE \eqref{eq:PDE} hyperbolisch, d.h. $B^2-AC>0$, also zwei unabhägige Scharen von Charakteristiken. Wähle $\xi, \eta$ so, dass $\Cc_\gamma, \Gc_\gamma$ Charakteristiken sind, d.h.
        \begin{align*}
           Q(\xi_x,-\xi_y) = 0 = Q(\eta_x,-\eta_y)
        \end{align*}
        \begin{align} %1.2.16
            A(\xi_x)^2 + B\xi_x\xi_y + C \xi_y^2 = 0 = a\\
            \Leftrightarrow\\
            A(\eta_x)^2 + B\eta_x\eta_y + C\eta_y^2 = 0 = c
            \label{}
        \end{align}\\
        $\to$ kanonische Form ist 
        \begin{equation} %1.2.20
            b \Phi_{\xi\eta} + e = 0
            \label{}
        \end{equation}
        Andere Wahlmöglichkeit (Courant, Hilbert II Kap III, Par.1, p.124)\\
            $a=-c, b=0$
            $\to$ kanonische Form ist 
            \begin{equation}%1.2.20a
                a(\Phi_{\xi\xi}-\Phi_{\eta\eta}) + e = 0
                \label{}
            \end{equation}
            (Anm. $u_{tt} = \tilde{c} u_{xx}$ Wellengleichung)


        \item Sei PDE parabolisch, d.h. $B^2-4AC = 0$ d.h. eine Schar von Charakteristiken $\to B^2-4AC = 0$, o.B.d.A. $AC\neq 0, A\neq 0 \to$ einzige Lösung von \ref{1.2.13} ist $\frac{dy}{dx} = \frac{B}{2A}$. Eine Schar von Charakteristiken $\Cc_\gamma: Q(\xi_x, -\xi_y)=0=a$. Mit \eqref{1.2.14} ist $b^2-\underset{=0}{4ac} = \underset{=0}{(B^2-4AC)}\underset{>0}{(det T')^2} = 0 \Rightarrow b=0$.\\
            $\to$ kanonische Form ist
            \begin{equation} %1.2.21
                c\Phi_{\eta\eta} + \underset{\text{neues } e}{e} = 0
                \label{}
            \end{equation}


            \item PDE elliptisch, $B^2-4AC<0$ keien Charakteristiken. Geeignete Koordinatentransformation erlauben es, gemischte Terme zum Verschwinden zu bringen\\
                $\to$ kanonische Form ist 
                \begin{equation} %1.2.22
                    a_1\Phi_{\xi\xi} + a_2 \Phi_{\eta\eta} + e = 0
                    \label{}
                \end{equation}
                mit $a_1,a_2 > 0$
\end{enumerate}


\section{Klassifizierung quasilinearere PDEs 2.Ordnung im $\R^d$ (vorher: d=2)} %I.3
Verallgemeinern der Klassifikation mittels Eigenwerten der Koeffizientenmatrix im PDE $\to$ Skript

\section{Einige grundlegende Modellgleichungen} %I.4
\subsection{Die Poisson-/Laplacegleichung}
Auf einem Gebiet $\Omega \subset \R^2$  betrachte die PDE 
\begin{equation} %1.4.1
    L[u] := -\Delta u = f
    \label{eq:laplace}
\end{equation}

, wobei $\Delta:= \frac{\partial^2}{\partial x^2} + \frac{\partial^2}{\partial y^2}$ Laplace-Operator, gegeben $f=f(x,y)$ auf $\Omega$, gesucht: $u=u(x,y), u: \Omega\to\R$\\

Da \eqref{eq:laplace} bereits in kanonischer Form \eqref{1.2.22} $\Rightarrow$ \eqref{eq:laplace} is elliptisch\\

Physikalischer Hintergrund: stationäre (zeitunabhängige) Probleme, z.B. Elastizität, Statik, stationäre Strömungsproblem

Beispiele:
\begin{itemize}
    \item $f \equiv 0: \Delta u= 0$ Potentialgleichung ($\to$ Geodäsie)\\
        (physikalisch: $u$ ist elektrostatisches Potential eines elektrischen Feldes $E=-\nabla u$)
    \item Elastizität: $-\Delta u = f$: $u$ beschreibt Auslenkung einer dünnen Membran, auf die durch $f$ ausgeübte Kräfte wikren.
    \item \eqref{eq:laplace} sind Eulergleichungen zu Variationsaufgaben wie z.B. das Plateauproblem
\end{itemize}

\begin{beispiel}[Beispiel einer Variationsaufgabe: Das Plateauproblem] %1.4.2
    (Berühmtes Problem aus Geometrie und Analysis [Minimalflächen]). Gesucht ist eine Fläche  $\Fc\subset\R^3$, die von einer gegebenen Kurve $\Cc$ berandet ist, wobei der Flächeninhalt $\mu(\Fc)$ minimal sein soll, d.h.
    \[
        \mu(\Fc) \overset{!}{=} min \text{ und } \partial\Fc = \Cc
    \]
    $\Fc$ hinreichend regulär $\to$ Problem ist äquivalent dazu, dass 
    \[
        H=0 \text{ auf } \Fc, \quad \partial\Fc=\Cc \quad \text{(mittlere Krümmung der Fläche)}
    \]

    Literatur: [J.-H. Eschenberg, J. Jost, Differentialgeometrie und Minimalflächen, 2. Aufl. 2007, Springer]\\

    Flächen mit $H=0$ heißen Minimalflächen.\\
    Speziell: ist $\Fc$ als $u=u(x,y)$ gesucht $\to$ das Minimalflächenproblem lautet für $\Omega\subset\R^2$
    BILD
    
    \begin{equation}%(1.4.2a)
        \underbrace{\int_{\Omega} \sqrt{1 + u_x^2 + u_y^2} dx dy}_{\text{Flächeninhalt}} \overset{!}{=} min, \quad u=g \text{ auf } \partial\Omega \text{ gegeben}
        \label{eq:minflach}
    \end{equation}

    Hinreichend reguläre Situation $\to$ \eqref{eq:minflach} äquivalent zu nichtlinearer PDE
    \begin{equation} %1.4.2b
        \left.
        \begin{aligned}
            \frac{\partial}{\partial x} \left( \frac{u_x}{\sqrt{1+u_x^2u_y^2}} \right) +  \frac{\partial}{\partial y} \left( \frac{u_y}{\sqrt{1+u_x^2+u_y^2}} \right) &= 0 \quad \text{ in } \Omega\\
            u &= g \quad { auf } \partial\Omega
        \end{aligned}
    \right\}
        \label{}
    \end{equation}
    Euler-Lagrange-Gleichungen zu \eqref{eq:1.4.2a}

    Lagrange[1762]: \eqref{eq:1.4.2b} ist Spezialfall für mehrdimensionale Variationsprobleme. 
    Plateau[Physiker, 19. Jh.]: Drahtrahmen $\Cc$ in Seifenlösung getaucht und zahlreiche Mathematiker im 19. Jh., Durchbruch erst im 20. Jh. mit modernen Methoden der Variationsrechnung. Exakte Lösung nur in Spezialfällen!

    Vereinfachung von \eqref{eq:1.4.2a} durch Linearisierung mittels Taylorentwicklung für kleine Werte von $u_x,u_y: \sqrt{1+z} = 1 + \frac{z}{2} + O(z^2) \to$ ersetze Integranden in \eqref{eq:1.4.2a} durch quadratischen Ausdruck $\Rightarrow$
    \begin{equation} %1.4.2c
        \begin{aligned}
            J(v) := \frac{1}{2}\int_{\Omega} (v_x^2+v_y^2) dx dy & \overset{!}{=} min\\
            v & = g \text{ auf } \partial\Omega
        \end{aligned}
        \label{eq:1.4.2c}
    \end{equation}
    Alternative Darstellung: $\underset{v, v=g \text{ auf } \partial\Omega}{min}$
    $J:\Cc^1(\Omega) \to \R$ ist ein Funktional.

    [Prinzipien der Variationsrechnung, R. Courant, D. Hilbert, Methoden der math. Physik I, Springer 1993]\\

    Minimierung (d.h. $\delta J(u) = 0$, sprich: 1. Variation von $J$ zu Null setzen) liefert
    \begin{equation} %1.4.2d
        \left.
            \begin{aligned}
                \Delta u & = 0 \quad \text{ in } \Omega\\
                u & = g \quad \text{ auf } \partial\Omega
            \end{aligned}
        \right\}
        \label{eq:eulerlagrange}
    \end{equation}
    (Euler-Langrange-Gleichung für lineareisierte Plateauprobleme)

    Variationsaufgaben spielen in dieser Vorlesung eine wichtige Rolle, daher Herleitung der PDE \eqref{eq:eulerlagrange} genauer:\\

    Dazu Annahme: es gibt eine Minimalösung $u\in \C^2(\Omega) \cap \C^0(\overline{\Omega})$ von \eqref{eq:1.4.2c} (klassische Lösung). Setze $D(v,w) := \int_{\Omega} (v_xw_x + v_yw_y) dx dy$ und $D(v):= D(v,v)$. Wegen $D$ (bi)linear und symmetrisch..
    \[
        D(v + \alpha w) = D(v) + 2\alpha D(v,w) + \alpha^2 D(w,w) \quad \forall v,w \in \Cc^2(\Omega) \cap \Cc^0(\overline{\Omega}) \forall \alpha\in\R
    \]
    Sei $v\in\C^1(\Omega)$ mit $v|_{\partial\Omega} = 0$. 

    1. Variation $\partial D(v) := \underset{\alpha\to0}{\lim} \frac{D(v+\alpha w) -D(v)}{\alpha}$
    J minimieren $\to$ $\delta D(v) = 0 \forall v$. 

    Wegen Annahme ($u$ löst \eqref{1.4.2c}) folgt 
    \[
        \partial D(u) = 0 = D(u,v)\\
        \forall v\in\Cc^1(\Omega) \quad \text{mit } v|_{\partial\Omega} = 0
    \]

    \[
        \begin{aligned}
            0 = D(u,v) & = \int_{\Omega} (u_x v_x + u_yv_y) dx dy\\
            & \underset{Green, part. Int.}{=} - \int_{\Omega} v(u_{xx} + u_{yy}) dx dy + \underbrace{\int_{\partial\Omega} v(u_x - u_y) ds}_{=0}\\
            & = - \int_{\Omega} (u_{xx}+u_{yy})v dx dy
        \end{aligned}
    \]
    Das \underline{Test}funktion $v$ beliebig
    $\Rightarrow u_{xx}+u_{yy} = 0 \text{ in} \Omega \Leftrightarrow \Delta u = 0$ 
\end{beispiel}

 %Zusätzlich zu \eqref{1.4.1} werden Nebenbedingungen benötigt, um die Lösungen eindeutig festzulegen. Wenn die Cauchy Daten auf einer Charakteristik gegeben sind, so lässt sich zumindest über die Taylorentwicklung keine explizite Lösung angeben. 

\eqref{1.4.1} elliptisch $\Rightarrow$ keine Charakteristiken, d.h. man kann Cauchy-Daten $\left( \left\{ u = g \text{ auf } \Cc\\ u_x = g \right. \right)$ auf einer beliebigen Kurve vorschreiben. Demnach kann man anhand eines (berühmten) Beispiels zeigen, dass \eqref{1.4.1} mit Cauchy-Daten nicht ``korrekt gestellt'' ist. 

\begin{beispiel} %1.4.4
    Das Problem $\Delta u = 0 $ auf $\Omega := {(x,y): x\geq 0}$ mit Cauchy-Daten $\left\{ u(0,y) = 0\\ u_x(0,y) = \frac{1}{M}sin My, \quad M\in\N \right.$ hat als Lösung 
    \[
        u(x,y) = \frac{1}{M}(sin My) \underbrace{\sinh Mx}_{\underset{Def.}{=} \frac{e^{Mx}-e^{-Mx}}{2}}
    \]
    Beachte für $M\to\infty$ ist $u_x(0,y)\to 0$ (glm. in $y$) \\

    $\Rightarrow$ man erwartet, dass ebenso $u(x,y)\to 0$ für $M\to\infty$\\

    aber $u(x,y) = \underbrace{\frac{1}{M^2}(\sin My)}_{\to 0}\underbrace{(\sinh Mx)}_{\to\infty}$ oszilliert für wachsendes $M$, approximiert aber \underline{nicht} den Grenzwert $0$. Also: die Lösung hängt \underline{nicht} stetig von den Anfangsdaten ab. (Also, selbst wenn es keine Charakteristiken gibt, und man Cauchy-Daten angeben kann, ist in diesem Fall die Eindeutigkeit der Lösung nicht gegeben).
Stattdessen: Vorgabe von \underline{Randdaten} liefert korrekt gestellte Probleme.
\end{beispiel}

\underline{Randwertaufgaben} (boundary value problems: bvp)

\begin{equation} %1.4.5
    \left.  
    \begin{aligned}
        -\Delta u & = f \quad \text{in } \Omega\\
        u\big|_{\partial\Omega} & = g \quad \text{auf } \partial\Omega
    \end{aligned}
\right\} Dirichletproblem
\label{eq:dirichlet}
\end{equation}

\begin{equation} %1.4.6
    \left.  
    \begin{aligned}
        -\Delta u & = f \quad \text{in } \Omega\\
        \frac{\partial u}{\partial \nu}\big|_{\partial\Omega} & = g \quad \text{auf } \partial\Omega
    \end{aligned}
\right\} Neumann Problem
\label{eq:neumann}
\end{equation}

\begin{equation} %1.4.7
    \left.  
    \begin{aligned}
        -\Delta u & = f \\
        u\big|_{\Gamma_1} & = g_1 \\
        \frac{\partial u}{\partial \nu}\big|_{\Gamma_2} &= g_2
    \end{aligned}
\right\} \text{Gemischtes Problem }\Gamma_1 \cup \Gamma_2 = \partial\Omega, \Gamma_1 \cap \Gamma_2 = \emptyset
\label{eq:gemischt}
\end{equation}


$\Omega$ Gebiet ist per Definition offen. Damit \eqref{eq:dirichlet}-\eqref{eq:gemischt} sinnvoll gestellt sind, muss man fordern, dass 
\[
    u\in \Cc^2(\Omega)\cap\Cc^0(\bar{\Omega})
\]
Man braucht nicht $u\in \Cc^2(\bar{\Omega})$.


\begin{definition}
    $u: \Omega\to\R$ heißt \underline{harmonisch} in $\Omega$, falls $u\in\Cc^2(\Omega)\cap\Cc^0(\bar{\Omega})$ ist und $\Delta u \equiv 0 \text{ in } \Omega$. 
    \underline{Maximumprinzip}: Eine in $\Omega$ harmonische nichtkonstante Funktion nimmt ihr Maximum und Minimum auf dem Rand an. \\
     Vergleichsprinzip $\Rightarrow$
    \begin{itemize}
        \item Eindeutigkeit
        \item stetige Abhängigkeit von Randdaten
        \item stetige Abhängigkeit von der rechten Seiten
    \end{itemize}
\end{definition}

Existenz \underline{klassischer} Lösungen (d.h. Lösungen $u$ mit $u\in\Cc^2(\Omega)\cap\Cc^0(\bar{\Omega})$ lässt sich häufig für Spezialfälle über Grundlösungen, Greensche Formeln, etc. zeigen. \\

In Kapitel ~\ref{chapterII} zeigen wir Existenz und Eindeutigkeit in geeigneten Rahmen (Variationsaufgaben) in geeigneten Räumen.

\subsection{Die Wärmeleitungsg-/Diffusionsgleichung} %1.4.2
(Vgl. Aufgabe 4, Zettel 3)\\
Einfachster Fall: eine Ortsvariable $x\in(0,l)$, Zeit $t\geq 0$. Die Lösung $u(x,t)$ sei die Temperatur eines dünnen Drahtes der Länge $l$ zur Zeit $t$ am Ort $x$. Die Temperaturverteilung wird beschrieben durch 
\setcounter{equation}{17}
\begin{equation} %1.4.18!!
    u_t = a^2 u_{xx} \quad x\in(0,l), t\geq0, a\neq 0 \text{konstant}
    \label{}
\end{equation}

Der Typ der PDE lässt sich mittel \eqref{eq:1.2.12} charakterisieren: $A=a^2, B=0=C \Rightarrow$ parabolisch. Eine Schar von Charakteristiken $a^2(\frac{dt}{dx})=0 \Leftrightarrow \frac{dt}{dx} = 0$ einzige Lösung, Kurven parallel zur $x$-Achse.\\

Ein Modell für mögliche Nebenbedingungen auf $\Omega:= \left\{ (x,t): x\in (0,l), t\geq 0 \right\}$

\begin{equation}
    \begin{aligned}
        u(x,0) &= g(x) \quad x\in (0,l) \text{ Anfangstemperatur zur Zeit } t=0\\
            u(0,t) &= h_1(t) \quad t\geq 0\\
            u(l,t) &= h_2(t) \quad t\geq 0\\
    \text{Randbedingungen}
\end{aligned}
    \label{}
\end{equation}
gesucht Temperaturentwicklung $u(x,t)$

Weiteres Beispiel: Finanzmathematik, speziell die Optionsbewertung. Man erhält eine parabolische Differentialgleichung, die Black-Scholes-Gleichung. Etwas zur \underline{Optionspreisbewertung} (Option pricing).\\
Aktie (stoch.): $S(t)$. Man interessiert sich für Optionen. Dies ist ein abgeleitetes Finanzprodukt (Derivat). Der faire Preis einer Option ist zu bestimmen. Black und Scholes haben eine deterministische PDE ohne stochastisches Modell angegeben. Sie ist parabolisch und eine exakte Lösung konnte explizit (bei europäischen Optionen) angegeben werden. 

 %\section{Die Wellengleichung (wave equation)} %I.4.3
Physikalisches Modell: Schwingungsverlauf (oscillating string) einer eingespannten Saite der Länge $l$ von vernachlässigbarer Dicke und eine äußere Kraft $f$, die die Saite zum Schwingen bringt. 

\begin{equation} %1.4.28
    u_{tt} - c^2u_{xx} = f(x,t) \quad u=u(x,t), c\neq 0
    \label{}
\end{equation}

Lösung ist eine Auslenkung (displacement) der Saite an der Stelle $x$ zur Zeit $t$. Vergleich: Diffusionsgleichung, dort: $u_t$ statt $u_{tt}$.\\

Anfangsbedinungen 
\begin{equation} %1.4.29
    \label{}
    \begin{aligned}
        \left.
            u(x,0) = g(x)
            u_t(x,0) = h(x) 
        \right\} \text{gegeben } x\in(0,l)
    \end{aligned}
\end{equation}

Randbedingungen
\begin{equation} %1.4.30
    u(0,t) = u(l,t) = 0 \quad \text{Saite eingespannt}
    \label{}
\end{equation}

Typ der Gleichung: Charakt. Gleichung \eqref{eq:1.2.12} ist $A(\frac{dt}{dx})^2 -B(\frac{dt}{dx}) + C = 0$
$A=-c^2, B=0, C=1 \Rightarrow B^2-4AC = 4c^2 > 0 \Rightarrow$  hyperbolisch\\

2 Charakteristikenscharen mit Steigungen gegeben durch 
\begin{equation}
    -c^2 (\frac{dt}{dx})^2 + 1 = 0 \Leftrightarrow \frac{dt}{dx} = \underset{+}{-} \frac{1}{c}
    \label{}
\end{equation}

Beachte: Anfangsdaten (Cauchy-Daten) \eqref{eq:1.4.29} auf $x$-Achse, und $x$-Achse ist \underline{nicht} charakteristisch, d.h. Problem wohlgestellt. 


Transformation auf kanonische Form: (Charakt. $leftrightarrow$ Koordinatensystem) mittels
\begin{equation} %1.4.32
    \left\{
    \begin{aligned}
        \zeta(x,t) = x-ct\\
        \eta (x,t) = x+ct
    \end{aligned}
\right\} \Rightarrow \frac{\eta-\zeta}{2x} = t, \frac{\zeta+\eta}{2c} = x
\end{equation}

\[
    \begin{pmatrix}
        \zeta_x & \zeta_t\\
        \eta_x & \eta_t
    \end{pmatrix}
    =
    \begin{pmatrix}
        1 & -c\\
        1 & 0
    \end{pmatrix}
    = 2c \neq 0
\]

Nebenrechung: \[
    \frac{dt}{dx} = \underset{+}{-} \frac{1}{c}\\
    \frac{dx}{dt} = \underset{+}{-} c\\
    \Rightarrow x \underset{+}{-}ct = k
\]

$\Phi(\zeta,\eta) = \Phi(\zeta(x,t),\eta(x,t)) = u(x,t)$, einsetzen in PDE \eqref{eq:1.4.28} 
\begin{equation}%1.4.33
    4c^2\Phi_{\zeta\eta} = \tilde{f} \quad (\text{mit } \tilde{f}\equiv 0 \text{ für } f\equiv 0)
    \label{}
\end{equation}
(Das Produkt zweier unabhängiger Ableitungen)

\underline{Analytisch konstruierte Lösung der Wellengleichung auf $\R$ (mit $f\equiv 0$)}\\
Betrachte die homogene PDE
\begin{equation}
    u_{tt} = c^2u_{xx} \quad \text{auf } \Omega=\R
    \label{}
\end{equation}
mit Cauchy-Daten
\[
    u(x,0) = g(x), \quad x\in\Omega\\
    u_t(x,0)=h(x),\quad x\in\Omega
\]

Aus \eqref{eq:1.4.33} $\Rightarrow$ alle Lösungen von \eqref{eq:1.4.33} haben die Form $\Phi(\zeta,\eta) = \alpha(\eta) + \beta(\zeta)$, wobei $\alpha(\cdot), \beta(\cdot)$ beliebig (genügend) glatte Funktionen \underline{einer} Veränderlichen $\rightarrow$ Rücktransformation \eqref{eq:1.4.32}
\begin{equation}
    \Rightarrow u(x,t) = \alpha(x+ct) + \beta(x-ct)
    \label{}
\end{equation}

Finde in \eqref{eq:1.4.35} Funktionen $\alpha,\beta$, so dass die Anfangsbedinungen \eqref{eq:1.4.29} erfüllt sind. Wir erhalten dann mit $t=0$:
\begin{equation}
    \tag{(*)}
    \begin{aligned}
        u(x,0) \underset{=}{\eqref{eq:1.4.35}} \alpha(x) + \beta(x) &\underset{=}{\eqref{eq:1.4.29}} g(x)\\
        u_t(x,0) = c\alpha'(x)-x\beta'(x) &= h(x)
    \end{aligned}
\end{equation}

Sei $H$ Stammfunktion von $h$, d.h. $H(x) = \int_{0}^{x}h(z) dz + \tilde{x}$. Folgende Wahl von $\alpha$ und $\beta$ lösen dann die Gleichung \eqref{eq:(*)}:
\[
    \alpha(x) := \frac{1}{2} g(x) + \frac{1}{2c} H(x)\\
    \beta(x) := \frac{1}{2} g(x) - \frac{1}{2c} H(x)
\]

Einsetzen in \eqref{eq:1.4.35} liefert

\begin{equation} %1.4.36
    \begin{aligned}
        u(x,t) &= \frac{1}{2} \left( g(x+ct) + g(x-ct) \right) + \frac{1}{2c} \left( H(x+ct)-H(x-ct) \right)\\
        & = \frac{1}{2} \left( g(x+ct) + g(x-ct) \right) + \frac{1}{2c} \int_{x-ct}^{x+ct} h(z) dz
    \end{aligned}
\end{equation}

das Bemerkenswerte dieser Gleichung ist, dass sie \underline{nur} Werte zur Zeit $t=0$, und zwar die Cauchy-Daten $g,h$, ausgewertet an verschobenen Argumenten (d'Alembertsche Formel).

\underline{Interpretation der Lösung}\\
In der Darstellung \eqref{1.4.36} hängt die Lösung \underline{nur} von den Daten im Interval $[x-ct,x+ct]$ ab. Der \underline{Abhängigkeitsbereich} wird von den beiden Cahrakteristiken durch $(x,t)$ begrenzt.
[Bild in dem der Abhängigkeitsbereich dargestellt wird]\\

Speziell für $h\equiv 0$ gilt:
$\Rightarrow$ Lösung ist durch Superposition von zwei Mustern (durch $g$ gegeben), die sich mit \underline{konstanter} Geschwindigkeit auseinanderbewegen.

Vergleich des Wurf eines Steines in einen See. Erst mit voranschreitender Zeit kann man Wellen an einem weiter entfernten Ort sehen (das Dreieck wird für größeres/höheres $t$ größer).

Insgesamt folgt \underline{endliche} Ausbreitungsgeschwindigkeit von Signalen bei der Wellengleichung, die nicht abklingeln (müssen). Insbesondere Störungen in den Anfangsdaten setzen sich mit \underline{konstanter} Geschwindigkeit fort.

 %\input{Vorlesung08}
 %\input{Vorlesung09}
 %\input{Vorlesung09}
 %\input{Vorlesung10}
 %\begin{definition}
    $v\in L_2(\Omega)$ possesses a weak derivates $D^{\alpha}v:= \frac{\partial^{|\alpha|}v}{\partial x^\alpha} \inL_2(\Omega), (\Omega\subset\R^d)$ if 
    \begin{equation}
        \left( \xi, \frac{\partial^{|\alpha|}v}{\partial x^\alpha} \right)_{L_2(\Omega)} = (-1)^{|\alpha|}\left( \frac{\partial^{|\alpha|}\xi}{\partial x^\alpha},v \right)_{L_2(\Omega)} \forall \text{ test functions} \xi\in \Cc^\infty_0(\Omega)
        \label{}
    \end{equation}
\end{definition}

\begin{beispiel}
    $v(x)=|x|, \quad x\in [-a,a]$
\end{beispiel}

Für $k\in\N_0:$
\[
    H^k(\Omega) := \left\{ v\in L_2(\Omega): \frac{\partial^{|\alpha|}}{\partial x^\alpha}v \in L_2(\Omega),\quad \forall |\alpha|\leq k \right\}
\]
(alle schwachen Ableitungen bis Ordnung $k$ in $L_2(\Omega)$)

\begin{satz}
    $H^k(\Omega)$ bildet einen Hilbertraum mit dem Skalarprodukt 
    \[
        (v,w)_{H^k(\Omega)} := \sum_{|\alpha|\leq k}\left( D^\alpha v, D^\alpha w\right)_{L_2(\Omega)}
    \]

    der induzierten Sobolevnorm
    \[
        \|v\|^2_{H^2(\Omega)} := \sum_{|\alpha|\leq k} \|D^\alpha v\|^2_{L_2(\Omega)}
    \]
    und Halbnorm
    \[
        |v|^2_{H^k(\Omega)} := \sum_{|\alpha|=k} \|D^\alpha v\|^2_{L_2(\Omega)}
    \]
    (Halbnorm, da $|v|_{H^k(\Omega)}=0 \Rightarrow v\equiv$, z.B. $v\equiv konst. \neq 0$)
\end{satz}

Mit Hilfe der Fouriertransformation lassen sich auch Sobolevräume $H^s$ für $s\in\R_+$ definieren ($H^{1/2},\cdots$ auf $\R^d$; ebenfalls über Waveletcharakterisierungen $\to$ nächstes Semester)

\begin{beispiel}
    Die charakteristische Funktion der B-Splines $N_1$ liegt im Soboleraum $H^s([0,1])$ für $0\leq s<\frac{1}{2}$. Die Hutfunktionen $N_2$ liegen ebenfalls in $H^s([0,2])$ für jedes $0\leq < \frac{3}{2}$.
\end{beispiel}

\begin{lemma}[Lemma von Sobolev (Einbettungssatz)] %2.2.1
    Es gilt $H^s(\Omega) \subset \Cc^k(\Omega)$ für $k\in\N_0, \quad s> k + \frac{d}{2}$.
    \label{}
\end{lemma}

\begin{definition}
    Mit $H^k_0(\Omega)$ bezeichnet man die Vervollständigung von $C^\infty_0(\Omega)$ bzgl. der $\|\cdot\|_{H^k(\Omega)}$ Norm, oder $\bar{\Cc^\infty_0(\Omega)}^{H^k(\Omega)} = H^k_0(\Omega)$.
    Vervollständigung heißt: Sei $(v_n)_{n\in\N} \in\Cc^\infty_0(\Omega)$ beliebige Cauchy-Folge $\Rightarrow$ für $v$ mit 
    \[
        \lim_{n\to\infty} \|v_n-v\|_{H^k(\Omega)} = 0
    \]
    ist $v\in \bar{C^\infty_0(\Omega)}^{H^k(\Omega)}$.

    Speziell:
    \[
        H^1_0(\Omega) = \left\{ v\in H^1(\Omega) : v|_{\partial\Omega} = 0 \right\}
    \]
    (entsprechend für $H^i_0$ verschwinden die $i$-ten Ableitungen am Rand)
\end{definition}

Für elliptische PDE ein ausgesprochen nützliches Hilfsmittel:
\begin{satz}[Poincare-Friedrichs-Ungleichung] %2.2.2
    Für $\Omega\subset\R^d$ offen und beschränkt gilt 
    \[
        \|v\|_{L_2(\Omega)} \leq c |v|_{H^1(\Omega)} \quad \forall v\in H^1_0(\Omega)
    \]
    (Die Funktionswerte lassen sich durch die ersten Ableitungen abschätzen - aber auch nur, wenn die Funktionen am Rand verschwinden - trivialerweise gilt: $\|v\|_{L_2(\Omega)}\leq \|v\|_{H^2(\Omega)}$)
\end{satz}

\begin{beweis}
    Nach Definition ist $\Cc^\infty_0(\Omega)$ dicht in $H^1_0(\Omega)$, braucht die Ungleichung nur für $v\in\Cc^\infty_0(\Omega)$ gezeigt werden.
    Betrachte speziell den Fall $d=1:$ Sei $\Omega = (a,b)\subset \R$. Sei $v=0$ auf $\R\without \Omega$ ($\Rightarrow v=0$ auf $\partial\Omega$). Dann gilt für $x\in\Omega$:
    \[
        |v(x)|^2 \underset{0-Addition}{&=} |v(x)-v(a)|^2 = \left| \int_{a}^{x} (1) v'(t) dt^2 \right|\\
        \underset{\text{Hölder}}{&\leq} \|1\|_{L_2(a,x)} \int_{a}^{x} (v'(t))^2 dt\\
        &\leq (b-a) \underbrace{\int_{a}^{b} (v'(t))^2dt}_{=|v|^2_{H^1(\Omega)}}
        \intertext{Integration beider Seiten bzgl. $x$}
        \Rightarrow \|v\|^2_{L_2(a,b)} \underset{Def.}{=} \int_{a}^{b} |v(x)|^2 dx\\
        \underset{vorherige\\Abschätzung}{\leq} \int_{a}^{b} dx (b-a) |v|^2_{H^1(\Omega)}\\
        = (b-a)^2 |v|^2_{H^1(\Omega)}
    \]
    Für Raumdimension $d>1$ integeriere über restliche Koordinaten.
\end{beweis}


\begin{bemerkung}
    Die Poincare-Friedrichs-Ungleichung gilt auch noch für Funktionen $v\in H^1_\gamma(\Omega):=\left\{ v\in H^1(\Omega): v|_\gamma = 0, \gamma\subet \partial\Omega, \nu_{d-1}(\gamma)>0 \right\}$
\end{bemerkung}

\begin{korollar}
    \[
        \|v\|_{H^k(\Omega)} \tilde |v|_{H^k(\Omega)}
    \]
    für alle $v\in H^k_0(\Omega), k\in\N$,  (Sobolevnorm und Halbnorm sind äquivalent auf $H^k_0(\Omega)$)
\end{korollar}

\begin{definition}
    Zwei Normane $\|\cdot\|_1, \|\cdot\|_2$ auf Vektorraum $V$ heißen äquivalent, $\|\cdot\|_1 \tilde \|\cdot\|_2$, falls es Konstanten $0<c_1 \leq c_2 < \infty$ unabhängig von $v$ gibt mit 
    \[
        c_1\|v\|_1 \leq \|v\|_2 \leq c_2 \|v\|_1 \quad \forall v\in V.
    \]

    Gilt nur $\|v\|_2 \leq c \|v\|_1$, so schreiben wir $\|v\|_2 \lessym \|v\|_1$
\end{definition}

\begin{definition}
    $X,Y$ seien Banachräume mit $X\subset Y$ mit $\|\cdot\|_X, \|\cdot\|Y$. Gilt $\|\x\|_Y \lessym \|x\|_X \quad \forall x\in X$, so heißt $X$ \underline{stetig} in $Y$ \underline{eingegebettet}, ``$X\hookrightarrow Y$''. Wenn zusätzlich $X$ dicht in $Y$, dann heißt es $X$ stetig und dicht in $Y$ eingebettet. 

\end{definition}

\begin{beispiel}
    Es gilt $H^k(\Omega) \hookrightarrow L_2(\Omega), H^k_0(\Omega) \hookrightarrow L_2(\Omega)$ ($H^0(\Omega) = L_2(\Omega) = H^0_0(\Omega)$)
\end{beispiel}
Nach Definition ist $H^k(\Omega) \subset \L_2(\Omega), v\in H^k(\Omega): \|v\|_{L_2(\Omega)} \lessym \|v\|_{H^k(\Omega)}$

\begin{definition} %2.2.3
    Sei $X$ ein normierter Raum mit $\|\cdot\|_X$ über $\R$. Sei $z: X\to\R$ eine beschränkte (=stetige) lineare Abbildung (auch lineares stetiges Funktionalt genannt), d.h. es gilt für die Operatornorm 
    \[
        \|z\|_{X\to\R} := \sub_{v\in X} \frac{|z(v)}{\|v\|_X} < \underbrace{\infty}_{``=beschränkt,stetig''}
    \]
    Der Dualraum $X'$ von $X$ ist die Menge \underline{aller} beschränkten Funktionale auf $X$. 
    Die Dualnorm ist obige Operatornorm, abgekürzt als 
    \[
        \|z\|_{X'} := \sup_{v\in X} \frac{|z(v)}{\|v\|_X} =: \sup_{v\in X} \frac{|<z,v>|}{\|v\|_X}
    \]<++>
\end{definition}<++>

 %\input{Vorlesung12}
 %\input{Vorlesung13}
%\input{Vorlesung14}
%\input{Vorlesung15}
%\input{Vorlesung16}
%\input{Vorlesung17}
%\input{Vorlesung18}
%\input{Vorlesung19}
%Weitere Eigneschaften von $A$ (z.B. Besetzungsstruktur) abhängig von der Wahl der Basis ${\phi_j}_{j=1}^n$ von $V_h$. Z.B. ${\phi_j}$ Tensor Produkte von B-Splines\\
$\to$ dünn besetzte/sparse Matrix
$\to$ im $\geq 2d$ iterative Löser $\to$ Verfahren der konjugierten Gradienten (CG-Verfahren)

\begin{beispiel} (CG-Verfahren)
    Annahme: $A$ symm.\ pos.\ def.\
    \begin{itemize}
        \item konstruiert Basis für $\R^n$, die bzgl. $a(\cdot,\cdot)$ orthogonal sind
        \item zusammen mit einem Abstiegsverfahren (z.B. dem Gradientenverfahren): Iteration mit steilstem Abstieg
    \end{itemize}
\end{beispiel}

Später: Konvergenzgeschwindigkeit des CG-Verfahrens (jedes iterativen Verfahrens) hängt von der spektralen Konditionszahl $\kappa_2(A)$ ab. (Hier: $\kappa_2(A) = \|A\|_2\|A^{-1}\|_2 \underset{=}{spd. \text{(Poweriteration)}} \frac{\lambda_{\max}(A)}{\lambda_{\min}(A)}$) (Vergl. Aufgabe 14g $\kappa_1(A)$)

Problem: Diskretisierung $h=2^{-J} \to A_J u_J = f_J$

\begin{beispiel}
    Hintergrund: PDE (V) auf $\infty-$dim.\ Raum:
    \[
        \|u_h -u\|_2 \to 0 \quad \text{mit } h\to 0\\ 
    \]
    Dies entspricht $u_J$ für $J\to\infty$
\end{beispiel}
\underline{Aber}: $\kappa_2(A_J) \tilde c(J)$

$A$ heißt Steifigkeitsmatrix (Strukturmechanik) bzw.\ gliedert sich 
\[
    a(v,w) = \underbrace{\int_{\Omega} (\nabla v \cdot \nabla w) dx}_{\text{Steifigkeitsmatrix}} + \int_{\Omega}vw dx
\]
Zur Beurteilung der \underline{Qualität} der Approximation, d.h.\ der Güter der Wahl des (endl.-dim) Funktionenraums $V_h$ werden Fehlerabschätzungen verwendet. Diese heißen \underline{A-Priori}-Fehlerabschätzung da man Fehler bereits vor jeglicher Rechnung abschätzen kann.
(Im Unterschied dazu bezeichnen A-Posteriori-Fehlerabschätzung z.B.\ $u_J$ ist Lösungsvektor für ein $u_H\in V_h$ und $u_{\tilde{J}}$ ist ein Lösungsvektor für ein $u_{\tilde{h}}\in V_{\tilde{h}}$. Schätze $\||u_J - u_{\tilde{J}}\||$ \& verfaeinere das Gitter dort, wo Koeffizienten groß sind)\\

Eine grundlegende Abschätzung liefert folgendes Lemma mit typischer Beweistechnik:

\begin{satz}[Cea-Lemma]
    Sei die Bilinearform $a(\cdot,\cdot)$ stetig und $V$-koerziv mit Konstanten $\alpha_1,\alpha_2$. Seien $u\in V$ mit $u_h\in V_h\subset V$ die Lösungen von $(V)$ bzw.\ $(V_h)$.\\
    Dann gilt: 
    \begin{equation}%2.4.6
        \underbrace{\|u-u_h\|_V}_{Diskretisierungsfehler} \leq \frac{\alpha_2}{\alpha_1} \inf_{v\in V_h} \|u-v\|_V
        \label{eq:cea}
    \end{equation}
    Dieses Lemma besagt also, dass die Genauigkeit der Lösung wesentlich davon abhängt, wie man den endlich-dim.\ Unterraum $V_h\subset V$ wählt.
\end{satz} 

\begin{bemerkung}
    Bei Differenzenverfahren (vgl. ODE, Advektionsgleichung, Ableitungen $\to$ Finite Differenzen) wird Approximation an $u$ nur auf diskretem Gitter berechnet. Approximationsabschätzungen dort i.d.R. punktweise. \\
    Bei Galerkin- (und Kollokations-) ansatz wird diskretisierte Lösung $u_h$ auf ganz $\Omega$ berechnet und ist dort direkt mit der exakten Lösung $u$ vergleichbar.
\end{bemerkung}

\begin{proof}[Beweis des Cea-Lemmas]
    Nach Definition von $u, u_h$ gilt 
    \[
        a(u,v) = \langle f,v \rangle \quad \forall v\in V\\
        a(u_h,v) = \langle f,v \rangle \quad \forall v\in V_h\\
    \]
    Wegen $V_h\subset V$ folgt durch Subtraktion 
    \begin{equation} %2.4.7
        a(u-u_h,v) = 0 \quad \forall v\in V_h
    \end{equation}
    (Wir bilden also bzgl.\ des Energie-inneren Produktes $a(\cdot,\cdot)$-orthogonale Projektion)
    Sei $w_h\in V_h$ beliebig.~\eqref{eq:2.4.7} für $v=w_h-u_h \in V_h$ \\
   \[
       \Rightarrow a(u-u_h, w_h-u_h) = 0
   \]
   Weiter ist 
   \[
       \begin{aligned}
           \alpha_1 \|u-u_h\|_V \underset{&\leq}{V-\text{Koerzivität}} a(u-u_h,u-u_h)\\
           &= a(u-u_h, u-w_h+w_h-u_h)\\
           &= a(u-u_h, u-w_h) + a(u-u_h,w_h-u_h)\\
           &\underset{\leq}{\text{Stetigkeit}} \alpha_2 \|u-u_h\|_V \|u-w_h\|V\\
           \Leftrightarrow \alpha_1\|u-u_h\|_V \leq \alpha_2\|u-w_h\|_v
       \end{aligned}
   \]
   Also folgt, wegen $w_h\in V_h$ beliebig~\eqref{eq:2.4.6}
\end{proof}

\begin{bemerkung} %2.4.8
    Ist  ${\phi_1, \dots, \phi_n}$ Orthonormalbasis des $V_h$ bzgl.\ $a(\cdot,\cdot)$, d.h.\ \\
    \[
        a(\phi_i,\phi_j) = \delta_{ij} \quad \forall i,j
    \]
    so hat jedes $v\in V_h$ die Darstellung
    \[
        v = \sum_{i=1}^{n} a(v,\phi_i)\phi_i
    \]
    Speziell für $u_h\in V_h$ als Lösung von ($V_h$)
    \[
        u_h:= \sum_{i=1}^{n} a(u, \phi_i)\phi_i
    \]
    hat man kein LGS zu lösen.
\end{bemerkung}<++>

%\input{Vorlesung21}
%\input{Vorlesung22}
%\input{Vorlesung23}
%\input{Vorlesung24}
%\input{Vorlesung25}
%\input{Vorlesung26}
%\input{Vorlesung27}
%\input{Vorlesung28}
%\input{Vorlesung29}

\printindex
\end{document}

