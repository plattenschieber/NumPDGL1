Weitere Eigneschaften von $A$ (z.B. Besetzungsstruktur) abhängig von der Wahl der Basis ${\phi_j}_{j=1}^n$ von $V_h$. Z.B. ${\phi_j}$ Tensor Produkte von B-Splines\\
$\to$ dünn besetzte/sparse Matrix
$\to$ im $\geq 2d$ iterative Löser $\to$ Verfahren der konjugierten Gradienten (CG-Verfahren)

\begin{beispiel} (CG-Verfahren)
    Annahme: $A$ symm.\ pos.\ def.\
    \begin{itemize}
        \item konstruiert Basis für $\R^n$, die bzgl. $a(\cdot,\cdot)$ orthogonal sind
        \item zusammen mit einem Abstiegsverfahren (z.B. dem Gradientenverfahren): Iteration mit steilstem Abstieg
    \end{itemize}
\end{beispiel}

Später: Konvergenzgeschwindigkeit des CG-Verfahrens (jedes iterativen Verfahrens) hängt von der spektralen Konditionszahl $\kappa_2(A)$ ab. (Hier: $\kappa_2(A) = \|A\|_2\|A^{-1}\|_2 \underset{=}{spd. \text{(Poweriteration)}} \frac{\lambda_{\max}(A)}{\lambda_{\min}(A)}$) (Vergl. Aufgabe 14g $\kappa_1(A)$)

Problem: Diskretisierung $h=2^{-J} \to A_J u_J = f_J$

\begin{beispiel}
    Hintergrund: PDE (V) auf $\infty-$dim.\ Raum:
    \[
        \|u_h -u\|_2 \to 0 \quad \text{mit } h\to 0\\ 
    \]
    Dies entspricht $u_J$ für $J\to\infty$
\end{beispiel}
\underline{Aber}: $\kappa_2(A_J) \tilde c(J)$

$A$ heißt Steifigkeitsmatrix (Strukturmechanik) bzw.\ gliedert sich 
\[
    a(v,w) = \underbrace{\int_{\Omega} (\nabla v \cdot \nabla w) dx}_{\text{Steifigkeitsmatrix}} + \int_{\Omega}vw dx
\]
Zur Beurteilung der \underline{Qualität} der Approximation, d.h.\ der Güter der Wahl des (endl.-dim) Funktionenraums $V_h$ werden Fehlerabschätzungen verwendet. Diese heißen \underline{A-Priori}-Fehlerabschätzung da man Fehler bereits vor jeglicher Rechnung abschätzen kann.
(Im Unterschied dazu bezeichnen A-Posteriori-Fehlerabschätzung z.B.\ $u_J$ ist Lösungsvektor für ein $u_H\in V_h$ und $u_{\tilde{J}}$ ist ein Lösungsvektor für ein $u_{\tilde{h}}\in V_{\tilde{h}}$. Schätze $\||u_J - u_{\tilde{J}}\||$ \& verfaeinere das Gitter dort, wo Koeffizienten groß sind)\\

Eine grundlegende Abschätzung liefert folgendes Lemma mit typischer Beweistechnik:

\begin{satz}[Cea-Lemma]
    Sei die Bilinearform $a(\cdot,\cdot)$ stetig und $V$-koerziv mit Konstanten $\alpha_1,\alpha_2$. Seien $u\in V$ mit $u_h\in V_h\subset V$ die Lösungen von $(V)$ bzw.\ $(V_h)$.\\
    Dann gilt: 
    \begin{equation}%2.4.6
        \underbrace{\|u-u_h\|_V}_{Diskretisierungsfehler} \leq \frac{\alpha_2}{\alpha_1} \inf_{v\in V_h} \|u-v\|_V
        \label{eq:cea}
    \end{equation}
    Dieses Lemma besagt also, dass die Genauigkeit der Lösung wesentlich davon abhängt, wie man den endlich-dim.\ Unterraum $V_h\subset V$ wählt.
\end{satz} 

\begin{bemerkung}
    Bei Differenzenverfahren (vgl. ODE, Advektionsgleichung, Ableitungen $\to$ Finite Differenzen) wird Approximation an $u$ nur auf diskretem Gitter berechnet. Approximationsabschätzungen dort i.d.R. punktweise. \\
    Bei Galerkin- (und Kollokations-) ansatz wird diskretisierte Lösung $u_h$ auf ganz $\Omega$ berechnet und ist dort direkt mit der exakten Lösung $u$ vergleichbar.
\end{bemerkung}

\begin{proof}[Beweis des Cea-Lemmas]
    Nach Definition von $u, u_h$ gilt 
    \[
        a(u,v) = \langle f,v \rangle \quad \forall v\in V\\
        a(u_h,v) = \langle f,v \rangle \quad \forall v\in V_h\\
    \]
    Wegen $V_h\subset V$ folgt durch Subtraktion 
    \begin{equation} %2.4.7
        a(u-u_h,v) = 0 \quad \forall v\in V_h
    \end{equation}
    (Wir bilden also bzgl.\ des Energie-inneren Produktes $a(\cdot,\cdot)$-orthogonale Projektion)
    Sei $w_h\in V_h$ beliebig.~\eqref{eq:2.4.7} für $v=w_h-u_h \in V_h$ \\
   \[
       \Rightarrow a(u-u_h, w_h-u_h) = 0
   \]
   Weiter ist 
   \[
       \begin{aligned}
           \alpha_1 \|u-u_h\|_V \underset{&\leq}{V-\text{Koerzivität}} a(u-u_h,u-u_h)\\
           &= a(u-u_h, u-w_h+w_h-u_h)\\
           &= a(u-u_h, u-w_h) + a(u-u_h,w_h-u_h)\\
           &\underset{\leq}{\text{Stetigkeit}} \alpha_2 \|u-u_h\|_V \|u-w_h\|V\\
           \Leftrightarrow \alpha_1\|u-u_h\|_V \leq \alpha_2\|u-w_h\|_v
       \end{aligned}
   \]
   Also folgt, wegen $w_h\in V_h$ beliebig~\eqref{eq:2.4.6}
\end{proof}

\begin{bemerkung} %2.4.8
    Ist  ${\phi_1, \dots, \phi_n}$ Orthonormalbasis des $V_h$ bzgl.\ $a(\cdot,\cdot)$, d.h.\ \\
    \[
        a(\phi_i,\phi_j) = \delta_{ij} \quad \forall i,j
    \]
    so hat jedes $v\in V_h$ die Darstellung
    \[
        v = \sum_{i=1}^{n} a(v,\phi_i)\phi_i
    \]
    Speziell für $u_h\in V_h$ als Lösung von ($V_h$)
    \[
        u_h:= \sum_{i=1}^{n} a(u, \phi_i)\phi_i
    \]
    hat man kein LGS zu lösen.
\end{bemerkung}<++>
