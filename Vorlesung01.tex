\chapter{Einleitung}
\section{Motivation, Themen dieser Vorlesung}
Thematik: Dynamische Prozesse aus den Natur-, Ingenieur-, Betriebs- und Sozialwissenschaften
\underline{Methodik:} 
\begin{enumerate}[I.]
        \item Mathematisch-physikalisches Modell: Abbildung Realität $\rightarrow$ Modell
        \item Theorie: Wohlgestelltheit des Problems: Existenz, Eindeutigkeit einer Lösung, stetige Abhängigkeit von Daten
        \item Numerik/Numerische Simulation: Berechnung einer approximativen Lösung mittels \underline{effizienter Algorithmen}
        \item Visualisierung der Ergebnisse
        \item Validierung der Ergebnisse mittels Realdaten, ggf. zurück zu I. und Nachbesserung des Modells (Z.B. konstante Volatiliät des Black Scholes Modells genügt der Realität nicht. Eine Anpassung des Modells in I. ist daher notwendig um ``sinnvolle'' Ergebnisse zu erhalten)
\end{enumerate}

In dieser Vorlesung: (und WissRech III-IV)
Betrachtung von Prozessen, die durch partielle Differentialgleichungen (partiel differential equations, PDE) beschrieben werden.
Stärker als bei (gewöhnlichen Differentialgleichungen) ODEs ist die Verzahnung von Modellierung, Theorie und Numerik

Ziel des \underline{Lösens einer PDE}: Bestimmung einer Funktion $u=u(x,y), (x,t)\in \R^3\times\R$ (mit $x$ Ort und $t$ Zeit, im Unterschied zu ODEs $y=y(t)$) aus Gleichungen, die partielle Ableitung von $u$ nach $x,t$ beinhalten. 

\section*{Grundlegende Definitionen}
Sei $\Omega \subset \R^d$ ein Gebiet (d.h. offen und zusammenhängend), $d\in\N$.
\begin{eqnarray*}
     x:=(x_1, \cdots, x_d)^T\in\R^d\\
     \alpha\in\Z_+^d \text{Multiindex}, |\alpha| := \alpha_1+\cdots+\alpha_d, x^\alpha:=x_1^{\alpha_1} \cdots x_d^{\alpha_d}\\
    \C^k(\Omega) := \left\{ f:\Omega\to\R: \frac{\partial^{|\alpha|}f}{\partial x^\alpha} \in \C^0(\Omega), \forall |\alpha|\leq k \right\}
\end{eqnarray*}

\begin{definition}%1.1.1
    Jede Funktion $u\in\C^k(\Omega)\times\C^l(\R)$, die zu gegebenem 
     $G: \R^{d+1}\times \R^{d+k \choose d)}\times\R^l \to \R^m$ die Gleichung 
     $G\left(x,t, \frac{\partial^{|\alpha|}u}{\partial x^\alpha}: |\alpha| \leq k, \frac{\partial^i u}{\partial t^i}, 1\leq i\leq l\right) = 0$(1.1.2) erfüllt, heißt klassische Lösung der durch (1.1.2) gegebenen PDE der Ordnung k im Ort und l in der Zeit.
\end{definition}

$d+k \choose d$: die Anzahl der Möglichkeiten aller Ableitungen

\begin{itemize}
    \item ``klassisch'': Lösung wird als genügend glatt vorausgesetzt
    \item ``partiell'': Differentialgleichung enthält partielle Ableitungen
    \item $k,l$: Ordnung der PDE bzgl. Ort und Zeit
\end{itemize}

\begin{beispiel}
    $\underset{(Raumdim)}{d=2}, \underset{(skalare Gleichung)}{m=1}, \underset{(Ordn. Ort)}{k=1}, \underset{(keine Zeitabl.)}{l=0}$
    \[
        G\left( x_1,x_2,u,\frac{\partial u}{\partial x_1}, \frac{\partial u}{\partial x_2} \right) := a(x) \frac{\partial u}{\partial x_1} + b(x) \frac{ \partial u}{\partial x_2} + c(x) u = 0
    \]
    Gesucht: $u=u(x_1,x_2), \tilde(u) = \tilde(u)(x_1,x_2,t) = u(x_1,x_2)v(t)$
\end{beispiel}

Beachte: wie bei ODEs benötigt man Nebenbedingungen, um aus Gesamtlösungsschar für (1.1.2) eine spezielle Lösung \underline{eindeutig} auszuwählen.

\begin*{definition}[Problem ``korekt gestellt'']
\begin{itemize}
    \item es gibt eind. Lösung
    \item Lösung hängt stetig von Nebenbedingungen/Daten ab (Anfangs-, Randbedingungen)
\end{itemize}

Definition (1.1.1) ist viel zu allgemein für WissRech. Dynamik der am häufigsten auftretenden physikalischen Prozesse aus zwei Klassen: 
\begin{enumerate}
    \item[(A)] Diffusion (Wärmeleitung, Black-Scholes-Gleichung)
    \item[(B)] Transport (Strömungsprobleme)
\end{enumerate}
\end{definition}
Für Numerik zunächst: lineare Prozesse oder linearisierte Prozesse (da nichtlineare PDEs gewöhnlicherweise quadratisch sind und in einem kleinen Gebiet linear approximiert werden können).

Weitere Klassifikationen:
\begin{enumerate}[(a)]
    \item irreversible Prozesse (Reaktion, Diffusion, Wärmeleitung)
    \item reversible Prozesse (Wellen)
    \item stationäre Prozesse (Variationsprobleme, Statik [Beispiel Stadion])
\end{enumerate}

Faustregel für mathematiche Sichtweise:
\begin{enumerate}[(a) $\to$]
    \item parabolische PDEs
    \item hyperbolische PDEs
    \item elliptische PDEs
\end{enumerate}

Wichtig für WissRech: Prozesstyp entscheidet \underline{Wahl} der Diskretisierungsmethode zur Entwicklung effizienter Algorithmen zur Berechnung einer approximativen Lösung.
Prinzipiell möglich: Differenzenverfahren (auf uniformen Gitter)
Probleme dabei: 
\begin{itemize}
    \item benötigt hohe Regularität der Lösung (Taylorentwicklung)
        \item of nicht physikalisch sinnvoll, z.B. Transportprobleme $\to$ Finite-Volumen-Verfahren
\end{itemize}
