\section{Die Wellengleichung (wave equation)} %I.4.3
Physikalisches Modell: Schwingungsverlauf (oscillating string) einer eingespannten Saite der Länge $l$ von vernachlässigbarer Dicke und eine äußere Kraft $f$, die die Saite zum Schwingen bringt. 

\begin{equation} %1.4.28
    u_{tt} - c^2u_{xx} = f(x,t) \quad u=u(x,t), c\neq 0
    \label{}
\end{equation}

Lösung ist eine Auslenkung (displacement) der Saite an der Stelle $x$ zur Zeit $t$. Vergleich: Diffusionsgleichung, dort: $u_t$ statt $u_{tt}$.\\

Anfangsbedinungen 
\begin{equation} %1.4.29
    \label{}
    \begin{aligned}
        \left.
            u(x,0) = g(x)
            u_t(x,0) = h(x) 
        \right\} \text{gegeben } x\in(0,l)
    \end{aligned}
\end{equation}

Randbedingungen
\begin{equation} %1.4.30
    u(0,t) = u(l,t) = 0 \quad \text{Saite eingespannt}
    \label{}
\end{equation}

Typ der Gleichung: Charakt. Gleichung \eqref{eq:1.2.12} ist $A(\frac{dt}{dx})^2 -B(\frac{dt}{dx}) + C = 0$
$A=-c^2, B=0, C=1 \Rightarrow B^2-4AC = 4c^2 > 0 \Rightarrow$  hyperbolisch\\

2 Charakteristikenscharen mit Steigungen gegeben durch 
\begin{equation}
    -c^2 (\frac{dt}{dx})^2 + 1 = 0 \Leftrightarrow \frac{dt}{dx} = \underset{+}{-} \frac{1}{c}
    \label{}
\end{equation}

Beachte: Anfangsdaten (Cauchy-Daten) \eqref{eq:1.4.29} auf $x$-Achse, und $x$-Achse ist \underline{nicht} charakteristisch, d.h. Problem wohlgestellt. 


Transformation auf kanonische Form: (Charakt. $leftrightarrow$ Koordinatensystem) mittels
\begin{equation} %1.4.32
    \left\{
    \begin{aligned}
        \zeta(x,t) = x-ct\\
        \eta (x,t) = x+ct
    \end{aligned}
\right\} \Rightarrow \frac{\eta-\zeta}{2x} = t, \frac{\zeta+\eta}{2c} = x
\end{equation}

\[
    \begin{pmatrix}
        \zeta_x & \zeta_t\\
        \eta_x & \eta_t
    \end{pmatrix}
    =
    \begin{pmatrix}
        1 & -c\\
        1 & 0
    \end{pmatrix}
    = 2c \neq 0
\]

Nebenrechung: \[
    \frac{dt}{dx} = \underset{+}{-} \frac{1}{c}\\
    \frac{dx}{dt} = \underset{+}{-} c\\
    \Rightarrow x \underset{+}{-}ct = k
\]

$\Phi(\zeta,\eta) = \Phi(\zeta(x,t),\eta(x,t)) = u(x,t)$, einsetzen in PDE \eqref{eq:1.4.28} 
\begin{equation}%1.4.33
    4c^2\Phi_{\zeta\eta} = \tilde{f} \quad (\text{mit } \tilde{f}\equiv 0 \text{ für } f\equiv 0)
    \label{}
\end{equation}
(Das Produkt zweier unabhängiger Ableitungen)

\underline{Analytisch konstruierte Lösung der Wellengleichung auf $\R$ (mit $f\equiv 0$)}\\
Betrachte die homogene PDE
\begin{equation}
    u_{tt} = c^2u_{xx} \quad \text{auf } \Omega=\R
    \label{}
\end{equation}
mit Cauchy-Daten
\[
    u(x,0) = g(x), \quad x\in\Omega\\
    u_t(x,0)=h(x),\quad x\in\Omega
\]

Aus \eqref{eq:1.4.33} $\Rightarrow$ alle Lösungen von \eqref{eq:1.4.33} haben die Form $\Phi(\zeta,\eta) = \alpha(\eta) + \beta(\zeta)$, wobei $\alpha(\cdot), \beta(\cdot)$ beliebig (genügend) glatte Funktionen \underline{einer} Veränderlichen $\rightarrow$ Rücktransformation \eqref{eq:1.4.32}
\begin{equation}
    \Rightarrow u(x,t) = \alpha(x+ct) + \beta(x-ct)
    \label{}
\end{equation}

Finde in \eqref{eq:1.4.35} Funktionen $\alpha,\beta$, so dass die Anfangsbedinungen \eqref{eq:1.4.29} erfüllt sind. Wir erhalten dann mit $t=0$:
\begin{equation}
    \tag{(*)}
    \begin{aligned}
        u(x,0) \underset{=}{\eqref{eq:1.4.35}} \alpha(x) + \beta(x) &\underset{=}{\eqref{eq:1.4.29}} g(x)\\
        u_t(x,0) = c\alpha'(x)-x\beta'(x) &= h(x)
    \end{aligned}
\end{equation}

Sei $H$ Stammfunktion von $h$, d.h. $H(x) = \int_{0}^{x}h(z) dz + \tilde{x}$. Folgende Wahl von $\alpha$ und $\beta$ lösen dann die Gleichung \eqref{eq:(*)}:
\[
    \alpha(x) := \frac{1}{2} g(x) + \frac{1}{2c} H(x)\\
    \beta(x) := \frac{1}{2} g(x) - \frac{1}{2c} H(x)
\]

Einsetzen in \eqref{eq:1.4.35} liefert

\begin{equation} %1.4.36
    \begin{aligned}
        u(x,t) &= \frac{1}{2} \left( g(x+ct) + g(x-ct) \right) + \frac{1}{2c} \left( H(x+ct)-H(x-ct) \right)\\
        & = \frac{1}{2} \left( g(x+ct) + g(x-ct) \right) + \frac{1}{2c} \int_{x-ct}^{x+ct} h(z) dz
    \end{aligned}
\end{equation}

das Bemerkenswerte dieser Gleichung ist, dass sie \underline{nur} Werte zur Zeit $t=0$, und zwar die Cauchy-Daten $g,h$, ausgewertet an verschobenen Argumenten (d'Alembertsche Formel).

\underline{Interpretation der Lösung}\\
In der Darstellung \eqref{1.4.36} hängt die Lösung \underline{nur} von den Daten im Interval $[x-ct,x+ct]$ ab. Der \underline{Abhängigkeitsbereich} wird von den beiden Cahrakteristiken durch $(x,t)$ begrenzt.
[Bild in dem der Abhängigkeitsbereich dargestellt wird]\\

Speziell für $h\equiv 0$ gilt:
$\Rightarrow$ Lösung ist durch Superposition von zwei Mustern (durch $g$ gegeben), die sich mit \underline{konstanter} Geschwindigkeit auseinanderbewegen.

Vergleich des Wurf eines Steines in einen See. Erst mit voranschreitender Zeit kann man Wellen an einem weiter entfernten Ort sehen (das Dreieck wird für größeres/höheres $t$ größer).

Insgesamt folgt \underline{endliche} Ausbreitungsgeschwindigkeit von Signalen bei der Wellengleichung, die nicht abklingeln (müssen). Insbesondere Störungen in den Anfangsdaten setzen sich mit \underline{konstanter} Geschwindigkeit fort.
